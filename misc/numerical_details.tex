\documentclass[11pt,a4paper,reqno,french,tikz]{amsart}
\usepackage[utf8]{inputenc} \usepackage[T1]{fontenc}
%\usepackage{babel} % français \usepackage[latin1]{inputenc}
\usepackage{amsthm,amsmath,amsfonts,amssymb,amsxtra,appendix,bookmark,dsfont,bm,mathrsfs,amstext,amsopn,mathrsfs,mathtools,comment,cite,hyperref,color,xcolor,cite}

\usepackage{stmaryrd}

%% Julia definition for Listings package (c) 2014 Jubobs
\usepackage{beramono,listings}
\lstdefinelanguage{Julia}%
  {morekeywords={abstract,break,case,catch,const,continue,do,else,elseif,end,export,false,for,function,immutable,import,importall,if,in,macro,module,otherwise,quote,return,switch,true,try,type,typealias,using,while},sensitive=true,alsoother={$},morecomment=[l]\#,morecomment=[n]{\#=}{=\#},morestring=[s]{"}{"},morestring=[m]{'}{'},}[keywords,comments,strings]
\lstset{language= Julia,basicstyle= \ttfamily,keywordstyle= \bfseries\color{blue},stringstyle= \color{magenta},commentstyle = \color{ForestGreen},showstringspaces = false,}
%% END Julia

%%%%%%%%%%%%%%%%%%%% Standard macros
% Environments
\newtheorem{theorem}{Theorem}[section]\newtheorem{definition}[theorem]{Definition}\newtheorem{lemma}[theorem]{Lemma}\newtheorem{example}[theorem]{Example}\newtheorem{proposition}[theorem]{Proposition}\newtheorem{corollary}[theorem]{Corollary}\newtheorem{conjecture}[theorem]{Conjecture}\newtheorem{remark}[theorem]{Remark}

% Mathbb
\let\C\relax\newcommand{\C}{\mathbb{C}}\newcommand{\Z}{\mathbb{Z}}\newcommand{\R}{\mathbb{R}}\newcommand{\N}{\mathbb{N}}\newcommand{\Q}{\mathbb{Q}}\newcommand{\bbM}{\mathbb{M}}

% Mathcal
\newcommand\cA{\mathcal{A}}\newcommand\cB{\mathcal{B}}\newcommand\cC{\mathcal{C}}\newcommand\cD{\mathcal{D}}\newcommand\cE{\mathcal{E}}\newcommand\cF{\mathcal{F}}\newcommand\cG{\mathcal{G}}\newcommand\cH{\mathcal{H}}\newcommand\cI{\mathcal{I}}\newcommand\cJ{\mathcal{J}}\newcommand\cK{\mathcal{K}}\newcommand\cL{\mathcal{L}}\newcommand\cM{\mathcal{M}}\newcommand\cN{\mathcal{N}}\newcommand\cO{\mathcal{O}}\newcommand\cP{\mathcal{P}}\newcommand\cQ{\mathcal{Q}}\newcommand\cR{\mathcal{R}}\newcommand\cS{\mathcal{S}}\newcommand\cT{\mathcal{T}}\newcommand\cU{\mathcal{U}}\newcommand\cV{\mathcal{V}}\newcommand\cW{\mathcal{W}}\newcommand\cX{\mathcal{X}}\newcommand\cY{\mathcal{Y}}\newcommand\cZ{\mathcal{Z}}

% Math operators
\DeclareMathOperator{\tr}{Tr}\DeclareMathOperator{\ran}{Ran}\DeclareMathOperator{\Span}{Span}\DeclareMathOperator{\Ker}{Ker}\DeclareMathOperator{\re}{Re}\DeclareMathOperator{\id}{id}\DeclareMathOperator{\im}{Im}\DeclareMathOperator{\dist}{dist}\let\div\relax\DeclareMathOperator{\div}{div}
\def\d{{\rm d}}

% Others
\renewcommand{\ge}{\geqslant}\renewcommand{\le}{\leqslant}

\newcommand{\intent}[1]{\llbracket #1 \rrbracket}

%%%%%%%%%%%%%%%%%%%% Macros Louis
% Delimiters
\newcommand{\pa}[1]{\left( #1 \right)} % ()
\newcommand{\acs}[1]{\left\{ #1 \right\}} % {}
\newcommand{\seg}[1]{\left[ #1 \right]} % []
\newcommand{\ab}[1]{\left|#1\right|} % ||
\newcommand{\ps}[1]{\left< #1 \right>} % <>
\newcommand{\proj}[1]{\big| #1 \big> \big< #1 \big|} % |u><u|
\newcommand{\nor}[2]{ \left| \! \left| #1 \right| \! \right|_{#2} } % ||.||

% Greek letters shortcuts
\newcommand\vp{\varphi} % φ
\def\eps{\varepsilon}\newcommand{\ep}{\varepsilon} % ε
\let\p\relax\newcommand{\p}{\psi} % ψ
\newcommand{\na}{\nabla} % ∇

% Others
\newcommand{\f}[2]{\frac{#1}{#2}} % fraction
\newcommand{\bul}{$\bullet$ \hspace{0.1cm}} % •
\newcommand{\mymax}[1]{\underset{\substack{#1}}{\text{\normalfont{max}}}\quad} % max
\newcommand{\mymin}[1]{\underset{\substack{#1}}{\text{\normalfont{min}}}\quad} % min
\newcommand{\mysup}[1]{\underset{\substack{#1}}{\text{\normalfont{sup}}}\quad} % sup
\newcommand{\myinf}[1]{\underset{\substack{#1}}{\text{\normalfont{inf}}}\quad} % inf
\newcommand{\ind}[1]{_{\textup{#1}}} % indice
\newcommand{\apo}[1]{#1''} % apostrophes "..."
\newcommand{\mat}[1]{\begin{pmatrix} #1 \end{pmatrix}} % matrices
\def\1{{\mathds{1}}}
\newcommand{\ketbra}[2]{\left| #1 \right> \left< #2 \right|}
\newcommand{\bbV}{\mathbb{V}}


\newcommand{\db}[1]{\left(\!\left( #1 \right)\!\right)}
\def\bX{{\mathbf X}}
\def\bG{{\mathbf G}}
\def\ba{{\mathbf a}}
\def\bb{{\mathbf b}}
\def\be{{\mathbf e}}
\def\bk{{\mathbf k}}
\def\bp{{\mathbf p}}
\def\bq{{\mathbf q}}
\def\br{{\mathbf r}}
\def\bx{{\mathbf x}}
\def\bmm{{\mathbf m}}
\def\bv{{\mathbf v}}
\def\bs{{\mathbf s}}
\def\by{{\mathbf y}}
\def\bn{{\mathbf n}}
\def\bA{{\mathbf A}}
\def\bH{{\mathbf H}}
\def\bK{{\mathbf K}}
\def\bR{{\mathbf R}}
\def\bS{{\mathbf S}}
\def\gR{{\mathfrak R}}

\def\bbI{{\mathbb I}}
\def\bbV{{\mathbb V}}
\def\bbW{{\mathbb W}}
\def\bbA{{\mathbb A}}
\def\bbB{{\mathbb B}}
\def\1{{\mathds{1}}}
\def\L{{\mathbb L}}

\newcommand{\dd}{\tfrac{d}{2}}
\newcommand{\sqom}{\sqrt{\ab{\Omega\ind{M}}}}

\def\lAngle{\langle\!\langle}
\def\rAngle{\rangle\!\rangle}
\def\ri{{\rm i}}
\def\rd{{\rm d}}
\def\re{{\rm e}}
\def\per{{\rm{per}}}
\newcommand{\ppa}[1]{\pa{\!\pa{#1}\!}_d}
\newcommand{\ept}{\varepsilon_\theta}
\def\bsigma{{\boldsymbol\sigma}}

%%%%%%%%%%%%%%%%%%%% End macros

\title[Numerics effective TBG]{Numerical computations for an effective model of twisted bilayer graphene}
\author[Louis Garrigue]{Louis Garrigue}
% \address{CERMICS, \'Ecole des ponts ParisTech, 6 and 8 av. Pascal, 77455 Marne-la-Vallée, France} 
% \email{louis.garrigue@enpc.fr}
% \date{\today}
\begin{document}
\maketitle


\begin{abstract}
We give here details on the implementation of the code corresponding to the article \cite{CanGarGon22b}, which proposes a derivation of effective moiré models from continuous Schrödinger operators.
\end{abstract}


\section{Standard monolayer}%
\label{sec:standard_monolayer}

We choose
\begin{align*}
	a_1 := a\mat{1/2 \\ -\sqrt{3}/2}, \qquad a_2 := a\mat{1/2 \\ \sqrt{3}/2}
\end{align*}
We define the matrix going from reduced to cartesian coordinates
\begin{multline*}
	\cM : \mathbb{T}^2 \simeq [0,1]^2 \rightarrow \Omega, \\
	\cM :=  \f a2 \mat{1 & 1 \\ -\sqrt{3} & \sqrt{3}} = \mat{a_1 & a_2}, \qquad  \cM^{-1} = \f 1a \mat{1 & -1/\sqrt{3} \\ 1 & 1/\sqrt{3}}
\end{multline*}
and
\begin{align*}
	2\pi \pa{\cM^{-1}}^* = \mat{a_1^* & a_2^*} = \f{2\pi}a \mat{1 & 1 \\ -\f{1}{\sqrt 3} & \f{1}{\sqrt 3}}= \sqrt{3} k_D \mat{\f{\sqrt 3}{2} & \f{\sqrt 3}{2} \\ -\f 12 & \f 12} =: S
\end{align*}
where $k_D := \f{4\pi}{3a}$.

\subsection{Dirac point}%
\label{sub:dirac_point}

We have
\begin{align*}
	\ab{a_j^*} = \sqrt{3} k_D, \qquad  K = -\f{a_1^* +a_2^*}{3}, \qquad a_1^*\cdot a_2^* = \f{3}{2} k_D^2, \qquad \ab{K} = k_D
\end{align*}

\subsection{From $q$ to $m_q$}%
\label{sub:from_q_to_m_q_}

Suppose you know $q$ in cartesian coordinates and you want to compute $m^q$, its reduced coordinates, that is $m^q a = q$, then
\begin{align*}
	m^q a = \mat{a_1^* & a_2^*} \mat{m^q_1 \\ m^q_2} = 2\pi \pa{\cM^{-1}}^* \mat{m^q_1 \\ m^q_2}
\end{align*}
so
\begin{align}\label{eq:inv_q}
\mat{m^q_1 \\ m^q_2} = \f{1}{2\pi} \cM^* q
\end{align}

\subsection{Fourier conventions}%
\label{sub:fourier_conventions}

We will manipulate functions which are $\Omega$-periodic in $\bx$, but not in $z$. We make the approximation that $L$ is large enough so that the $z$-periodized systems are equal. So now we consider that $f$ and $g$ are $L$-periodic in $z$, and $\int_\R \d z \simeq cst \int_{[0,L]} \d z$ so the Fourier transform is
\begin{align*}
	\boxed{\pa{\cF f}_{m,m_z} := \f{1}{\Gamma} \int_{\Omega\times [0,L]} e^{-i\pa{ma^* \bx + m_z \f{2\pi}{L} z}} f(\bx,z) \d \bx \d z}
\end{align*}
where $\Gamma := \sqrt{L\ab{\Omega}}$
and the reconstruction formula is
\begin{align}\label{eq:dec_four}
\boxed{f(\bx,z) =  \sum_{\substack{m \in \Z^2 \\ m_z \in \Z}}  \f{e^{i\pa{m \ba^* \cdot \bx + m_z \f{2\pi}L z}}}{\Gamma} \widehat{f}_{m,m_z}}
\end{align}
We define the scalar product
\begin{align*}
\ps{f,g} := \int_{\Omega\times [0,L]} \overline{f}g
\end{align*}
and compute Plancherel's formula
\begin{align}\label{eq:plancherel}
\ps{f,g} = \sum_{\substack{m \in \Z^2 \\ m_z \in \Z}} \overline{\widehat{f}_{m,m_z}} \widehat{g}_{m,m_z}.
\end{align}
Hence, as a verification, we test that the normalization of the $\widehat{u_j}$'s is the right one by checking that $\nor{u_j}{L^2}^2 = 1$ via \eqref{eq:plancherel}.

We implement the Fourier transforms
\begin{lstlisting}
myfft(a,B) = fft(a)*sqrt(B)/length(a)
myifft(a) = ifft(a)*length(a)/sqrt(B)
\end{lstlisting}
where $B = \Gamma^2 = L \ab{\Omega}$ in $3d$, $B = L$ in $1d$ in $z$, and $B=\ab{\Omega}$ in $2d$ in $(x,y)$. If $a_i = f(x_i)$ are the actual values of the functions, then $myfft(a)[m] \simeq \pa{\cF f}_{m-1}$ up to Riemann series errors.


\subsection{Rotation action}%
\label{sub:rotation_action}

We define $R_\theta := \mat{\cos \theta & -\sin \theta \\ \sin \theta & \cos \theta}$. We know that $R_{\f{2\pi}3} \pa{ma^* } = \pa{R_{\f{2\pi}3}^{\text{red}} m} a^*$ where
\begin{align*}
	R_{\f{2\pi}3}^{\text{red}} = S^{-1} R_{\f{2\pi}3} S =  \cM^* R_{\f{2\pi}3} \pa{\cM^*}^{-1} = \mat{-1 & -1 \\ 1 & 0}, \qquad R_{-\f{2\pi}3}^{\text{red}} = \mat{0 & 1 \\ -1 & -1}
\end{align*}
and
\begin{align*}
\cR_{\f{2\pi}3} f(x) =  \sum_m f_m e^{i m a^* \cdot R_{-\f{2\pi}3} x}= \sum_m f_m e^{i \pa{R_{\f{2\pi}3}^{\text{red}} m} a^* \cdot x} = \sum_m f_{R_{-\f{2\pi}3}^{\text{red}} m} e^{im a^* \cdot x}
\end{align*}



\subsection{Action of mirror}%
\label{sub:action_of_mirror}

We define $G := \text{diag } \pa{1,-1}$ and the action $\cG f(x) := f(Gx)$, we compute
\begin{align*}
	G^{\text{red}} = \cM^* G \pa{\cM^*}^{-1} = \mat{0 & 1 \\ 1 & 0}
\end{align*}




\section{Change of basis for getting $\Phi_j \in L^2_{\tau,\overline{\tau}}$}%
\label{sub:change_of_basis_for_getting_l_2__tau_tau}
Numerically, DFTK gives 
\begin{align*}
\phi, \p \in \Ker \pa{\cR_{\f{2\pi}{3}}-\tau} + \Ker\pa{\cR_{\f{2\pi}{3}}-\overline{\tau}}
\end{align*}
but we want to separate the spaces and obtain $\phi_1 \in \Ker \pa{\cR_{\f{2\pi}{3}}-\tau}$ so that $\phi_2(x,z) := \overline{\phi_1}(-x,z) \in \Ker\pa{\cR_{\f{2\pi}{3}}-\overline{\tau}}$, which existence is ensured by \cite{FefWei12}.

First we define
\begin{align*}
c := \nor{\pa{\cR_{\f{2\pi}{3}} -\tau}\phi_a}{L^2}^2, \qquad s := \ps{\pa{\cR_{\f{2\pi}{3}} -\tau}\phi_a,\pa{\cR_{\f{2\pi}{3}} -\tau}\phi_b}.
\end{align*}
Then we parametrize
\begin{align*}
\phi_1 = e^{i \alpha} \pa{\f s{\ab{s}} \cos \theta \phi_a + e^{i\beta} \sin \theta \phi_b}
\end{align*}
and we want $\pa{\cR_{\f{2\pi}{3}} -\tau}\phi_1 = 0$ which is equivalent to
\begin{align*}
\f s{\ab{s}}\cos \theta \pa{\cR_{\f{2\pi}{3}} -\tau}\phi_a + e^{i\beta} \sin \theta \pa{\cR_{\f{2\pi}{3}} -\tau}\phi_b =0
\end{align*}
and we take the scalar product with $\pa{\cR_{\f{2\pi}{3}} -\tau}\phi_a$ so that
\begin{align*}
\f c{\ab{s}}\cos \theta  + e^{i\beta} \sin \theta =0
\end{align*}
Now we necessarily have $e^{i\beta} = \pm$ so $\cos \theta = \mp \f {\ab{s}}c \sin \theta$ and finally using $\cos^2 + \sin^2 = 1$,
\begin{align*}
\ab{\cos \theta} = \f 1{\sqrt{1+ \pa{\f c{\ab{s}}}^2}}, \qquad \ab{\sin \theta} = \f 1{\sqrt{1+ \pa{\f {\ab{s}}c}^2}},
\end{align*}
and also choosing $\alpha = 0$ if $\cos \theta \ge 0$ and $\pi$ otherwise, which does not change anything, we have
\begin{align*}
\phi_1 = \f{s}{\ab{s}}\f 1{\sqrt{1+ \pa{\f c{\ab{s}}}^2}} \phi_a \pm \f 1{\sqrt{1+ \pa{\f {\ab{s}}c}^2}} \phi_b
\end{align*}
and $\phi_2(x) = \overline{\phi_1(-x)}$. By multiplying by $e^{-iKx}$, we also obtain
\begin{align*}
\boxed{u_1 = \f{s}{\ab{s}}\f 1{\sqrt{1+ \pa{\f c{\ab{s}}}^2}} u_a \pm \f 1{\sqrt{1+ \pa{\f {\ab{s}}c}^2}} u_b}
\end{align*}
and $u_2(x) = \overline{u_1(-x)}$.





\section{Computation of $V\ind{int}$}%
\label{sec:computation_of_vint_}

\subsection{Reduction of Fourier coefficients in $2d$ to $1d$}%
\label{sub:reduction_of_fourier_coefficients_in_2d_to_1d_}

In $1d$, the Fourier transform is
\begin{align*}
\pa{\cF h}_{m_z} := \f{1}{\sqrt L} \int_{[0,L]} e^{-im_z \f{2\pi}{L} z} h(z) \d z
\end{align*}
and the reconstruction formula is
\begin{align*}
	h(z) =  \f{1}{\sqrt L}\sum_{\substack{m_z \in \Z}}  e^{i m_z \f{2\pi}L z} \widehat{h}_{m_z}
\end{align*}

 We take a function $f$ in $3d$ and define its average
\begin{align*}
g(z) := \f{1}{\ab{\Omega}} \int_\Omega f
\end{align*}
and since
\begin{align*}
\widehat{f}_{0,m_z} = \f{1}{\Gamma} \int_\Omega f(x,z) e^{-i\f{2\pi}{L} m_z z} \d x \d z
\end{align*}
then
\begin{align*}
\widehat{g}_{m_z} = \f{1}{\ab{\Omega} \sqrt{L}} \int_{\Omega \times [0,L]} f(x,z) e^{-i\f{2\pi}{L} m_z z} \d x \d z = \f{\widehat{f}_{0,m_z}}{\sqrt{\ab{\Omega}}}
\end{align*}

\subsection{Computation}%
\label{sub:Computation}


For $\bs \in \Omega := [0,1] \ba_1 + [0,1] \ba_2$, we denote by $V^{(2)}_{\bs}$ the true Kohn-Sham mean-field potential for the configuration where the two sheets are aligned (no angle), but with the upper one shifted by a vector $\bs$. We set
\begin{align*}
	V_{\rm int, \bs}(z) &:= \f{1}{\ab{\Omega}} \int_{\Omega}  \left( V^{(2)}_{\bs}(\bx, z) - V(\bx, z + \dd) - V(\bx - \bs, z -\dd)   \right) \d \bx \\
	&= \f{1}{\ab{\Omega}} \int_{\Omega}  \left( V^{(2)}_{\bs}(\bx, z) - V(\bx, z + \dd) - V(\bx, z -\dd)   \right) \d \bx \\
    &= \f{1}{\ab{\Omega}^{\f 32}} \sum_{\substack{m \in \Z^2 \\ m_z \in \Z}}  \pa{ \widehat{\pa{V^{(2)}_{\bs}}}_{m,m_z} - \widehat{V}_{m,m_z}e^{i m_z \f{2\pi}{L} \f{d}{2}} -\widehat{V}_{m,m_z} e^{-i m_z \f{2\pi}{L} \f{d}{2}}} \\
    &\qquad \qquad \times \int_{\Omega} e^{i \pa{m \ba^* \cdot \bx + m_z \f{2\pi}{L} z}}\d \bx \\
    &= \f{1}{\sqrt{\ab{\Omega}}}\sum_{\substack{m_z \in \Z}} e^{i  m_z \f{2\pi}{L} z} \pa{ \widehat{\pa{V^{(2)}_{\bs}}}_{0,m_z} - 2 \widehat{V}_{0,m_z} \cos\pa{ m_z \tfrac{\pi d}{L} }}
\end{align*}
and we obtain the Fourier coefficients
\begin{align*}
\pa{\widehat{V_{\rm{int},\bs}}}_{m_z} =\f{1}{\sqrt{\ab{\Omega}}} \pa{\widehat{\pa{V^{(2)}_{\bs}}}_{0,m_z} - 2 \widehat{V}_{0,m_z} \cos\pa{ m_z \tfrac{\pi d}{L} }}
\end{align*}
We then compute
\begin{align*}
V\ind{int}(z) := \f{1}{\ab{\Omega}} \int_\Omega V_{\rm int, \bs}(z) \d \bs = \f{1}{N\ind{int}^2} \sum_{\substack{s_x,s_y \in \intent{1,N\ind{int}}}} V^{\rm{array}}_{\rm int, (s_x,s_y)}(z)
\end{align*}
and finally obtain the Fourier coefficients
\begin{align*}
\boxed{\pa{\widehat{V\ind{int}}}_{m_z} = \f{1}{N\ind{int}^2} \sum_{\substack{s_x,s_y \in \intent{1,N\ind{int}}}} \pa{\widehat{V_{\rm{int},\bs}}}_{m_z}}
\end{align*}
and we expect $V_{\rm{int},\bs}$ not to depend too much on $\bs$, that is we expect that the following quantity is small
\begin{align*}
	\delta_{V\ind{int}} &:= \f{\int_{\Omega\times \R} \ab{V_{\rm{int},\bs}(z) - V\ind{int}(z)}^2 \d \bs \d z}{\ab{\Omega} \int_{\R} V\ind{int}(z)^2 \d z} \\
&= \f{\sum_{m_z} \int_{\Omega} \ab{\pa{\widehat{V_{\rm{int},\bs}}}_{m_z} - \pa{\widehat{V\ind{int}}}_{m_z}}^2 \d \bs }{\ab{\Omega} \sum_{m_z}  \pa{\widehat{V\ind{int}}}_{m_z}^2} \\
&= \f{\sum_{s_x,s_y,m_z} \ab{\pa{\widehat{V}_{\rm{int},(s_x,s_y)}}_{m_z} - \pa{\widehat{V\ind{int}}}_{m_z}}^2 }{N\ind{int}^2 \sum_{m_z}  \pa{\widehat{V\ind{int}}}_{m_z}^2}
\end{align*}



\section{Effective potentials}%
\label{sec:effective_potentials}

We are now in $2d$ and $\ab{\Omega\ind{M}} = \ab{\Omega}$. 
We defined
\begin{align*}
\db{ f, g}^{\eta,\eta'}(\bX) :=   \int_{\Omega \times \R} \overline{f}\pa{x - \tfrac 12 \eta J\bX,z- \eta\dd} g\pa{x - \tfrac 12 \eta' J \bX, z- \eta'\dd} \d \bx \d z
\end{align*}
and
\begin{align*}
\boxed{\lAngle f, g \rAngle^{\eta,\eta'} := e^{i\f 12 \pa{\eta-\eta'} \bK \cdot J \bX}\db{ f, g}^{\eta,\eta'}}
\end{align*}
and in particular since $q_1 = J K$, then $\lAngle f, g \rAngle^{+-} = e^{-iq_1 x}\db{ f, g}^{+-}$. 



We have $\db{f,g}^{++} = \db{f,g}^{--} = \ps{f,g} = \sum_{m,m_z} \overline{\widehat{f}_{m,m_z}} \widehat{g}_{m,m_z}$. We also define, for $\alpha \in \{-1,1\}$,
\begin{align*}
C^\alpha_m := \sqom\sum_{m_z \in \Z} e^{ i2\alpha \f{d\pi}{L} m_z} \overline{\widehat{f}_{-\alpha m,m_z}} \widehat{g}_{-\alpha m,m_z}
\end{align*}
Using the Fourier decomposition \eqref{eq:dec_four} in $2d$, and using $a^*\ind{M} = J a^*$ and $ma^* \cdot JX = -m a\ind{M}^* \cdot X$, $\iota^\eta_{\eta'} := \f 12 \pa{\eta - \eta'}$, when $\eta \neq \eta'$,
\begin{align*}
	\db{f,g}^{\eta,\eta'} &=   \sum_{m \in \Z^2} e^{i\f 12 \pa{\eta-\eta'} ma^* \cdot J \bX} \sum_{m_z \in \Z} e^{i\pa{\eta-\eta'} \f{2\pi}{L} m_z \f{d}{2}} \overline{\widehat{f}_{m,m_z}} \widehat{g}_{m,m_z}\\
			      &=   \sum_{m \in \Z^2} e^{i ma^*\ind{M} \cdot J \bX} \sum_{m_z \in \Z} e^{i \iota^\eta_{\eta'} \f{2\pi}{L} m_z d} \overline{\widehat{f}_{-\iota^\eta_{\eta'} m,m_z}} \widehat{g}_{-\iota^\eta_{\eta'} m,m_z}\\
			      & = \sum_{m \in \Z^2} \f{e^{i ma^*\ind{M} \cdot J \bX}}{\sqom} C_{m}^{\iota^\eta_{\eta'}}
\end{align*}
To sum up,
\begin{align*}
	\boxed{\db{f,g}^{\eta,\eta'} = \delta_{\eta = \eta'} \ps{f,g} + \delta_{\eta \neq \eta'}\sum_{m \in \Z^2} \f{e^{i ma^*\ind{M} \cdot J \bX}}{\sqom} C_{m}^{\iota^\eta_{\eta'}}}
\end{align*}



For the potentials, we finally need to implement
\begin{multline*}
\bbW_{j,j'}^+ = \db{\overline{u}_j u_{j'}, V}^{+-}, \qquad \bbW_{j,j'}^- = \db{\overline{u}_j u_{j'}, V}^{-+},\\
\bbV_{j,j'} = \lAngle \pa{V+V\ind{int}} u_j , u_{j'} \rAngle^{+-}
\end{multline*}




\subsection{$\bbW$'s $V\ind{int}$ term}%
\label{sub:_bbw_s_vint_term}
We write $V\ind{int}(z) = \f{1}{\sqrt{L}} \sum_{m_z \in \Z} \widehat{V}\ind{int}^{m_z} e^{i \f{2\pi}L m_z z}$ hence
\begin{align*}
	\ps{u_j, V\ind{int} u_{j'}} &= \f{1}{L^{\f 32}} \sum_{\substack{m \in \Z^2 \\ m_z,m_z',M_z \in \Z}} \pa{\overline{\widehat{u}}_j}_{m,m_z} \pa{\widehat{u}_{j'}}_{m,m_z'}\pa{\widehat{V\ind{int}}}_{M_z} \int_z e^{iz \f{2\pi}{L} \pa{M_z + m_z'-m_z}} \\
& = \f{1}{\sqrt{L}} \sum_{\substack{m \in \Z^2 \\ m_z,m_z' \in \Z}} \pa{\overline{\widehat{u}}_j}_{m,m_z} \pa{\widehat{u}_{j'}}_{m,m_z'}\pa{\widehat{V\ind{int}}}_{m_z-m_z'} 
\end{align*}
and the matrix $M_{j,j'} := \ps{u_j, V\ind{int} u_{j'}}$ is such that $M^* = M$ and $M_{11} = M_{22}$.

In the function $\bbV(X) = \ps{u_j,V u_i}(X)$, when $V \rightarrow V+ V\ind{int}$, we have 
\begin{align*}
\widetilde{\bbV}(X) = \ps{u_j,(V+V\ind{int}) u_i}(X) = \bbV(X) + \ps{u_j,V\ind{int} u_i}
\end{align*}
but at the level of Fourier coefficients,
\begin{align*}
\widehat{\widetilde{\bbV}}_0 = \widehat{\bbV}_0 + \f{\ps{u_j,V\ind{int} u_i}}{\sqrt{\ab{\Omega}}}
\end{align*}
so when we add it to the Fourier Hamiltonian, we should not forget to divide by $\sqrt{\ab{\Omega}}$

\subsection{Adding a constant}%
\label{sub:adding_a_constant}
We have
\begin{align*}
g(x) := f(x) + c \qquad \iff \qquad \widehat{g}_0 = \widehat{f}_0 + \sqom c
\end{align*}

\subsection{Substracting the mean of $\bbW^+$}%
\label{sub:substracting_the_mean_of_bbw_}

To do this, we do it for a function $f$, 

\begin{align*}
	\frac{1}{\ab{\Omega}} \int_{\Omega\ind{M}} f = \f{\widehat{f}_{0}}{\sqrt{\ab{\Omega}}}
\end{align*}
hence
\begin{align*}
g(x) := f(x) -  \frac{1}{\ab{\Omega}} \int_{\Omega\ind{M}} f \qquad \implies \qquad \widehat{g}_0 = 0
\end{align*}


\subsection{$V\ind{int}^{3d}$}%
\label{sub:_v__int}

We have $V\ind{int}^{3d}(x,z) := V\ind{int}(z)$ hence $\pa{\widehat{V}\ind{int}^{3d}}_{m,m_z} = \sqom \pa{\widehat{V}\ind{int}}_{m_z}$




\section{Non Local term}%
\label{sec:non_local_term}

From the theoretical investigations, we have
\begin{align*}
F^{\eta,j,s}(\bX) := \int_{\R^3} \overline{\vp_{\text{Bl},s}(\by,z)} \Phi_j\pa{\by +\ba_s -2\eta J \bX,z-\eta d} \d \by \d z
\end{align*}
and
\begin{align*}
\bbW_{\text{nl},-1}^\eta\pa{\bX}_{jj'} := \f{v\ind{F}}{\ab{\Omega}} \sum_{s \in \{1,2\}} \overline{F^{\eta,j,s}(\bX)} F^{\eta,j',s}(\bX).
\end{align*}

Since $\vp_{\text{Bl},s}$ is localized, we periodize it and we make the approximation
\begin{align*}
	F^{\eta,j,s}(\bX) &\simeq \int_{\Omega \times [0,L]} \overline{\vp_{\text{Bl},s}(\by,z)} \Phi_j\pa{\by +\ba_s -2\eta J \bX,z-\eta d} \d \by \d z \\
&=\int_{\Omega \times [0,L]} \overline{\vp_{s}(\by,z)} u_j\pa{\by +\ba_s -2\eta J \bX,z-\eta d} \d \by \d z
\end{align*}
and we define $\vp$ such that $\vp_{\text{Bl},s} = e^{i \bK \by} \vp_s$, because it is $\widehat{\vp_s}$ which is stored by DFTK, so
\begin{align*}
	\vp_{s}(\by,z) &= \sum_{m, m_z} \f{e^{i\pa{m a^* \by + m_z \f{2\pi}{L}  z}}}{\Gamma} \widehat{\vp}_{s,m,m_z}, \\
	u_j(\by,z) &= \sum_{m, m_z} \f{e^{i\pa{m \by + \f{2\pi}{L} m_z z}}}{\Gamma} \widehat{\pa{u_j}}_{m,m_z}
\end{align*}
where $\bK$ is the Dirac point, thus
\begin{align*}
F^{\eta,j,s}(\bX) &=  \sum_{m, m_z} e^{i\pa{m \ba^* \pa{\ba_s - 2\eta J \bX} - \eta \f{2\pi}{L} m_z d}} \overline{\widehat{\vp}}_{s,m,m_z} \widehat{\pa{u_j}}_{m,m_z} \\
		  &=  \sum_{m, m_z} e^{i\pa{m \ba\ind{M}^* \pa{\f{1}{2} J \ba_s + \eta \bX} - \eta \f{2\pi}{L} m_z d}} \overline{\widehat{\vp}}_{s,m,m_z} \widehat{\pa{u_j}}_{m,m_z} \\
		  &=  \sum_{m, m_z} e^{i\pa{m \ba\ind{M}^* \pa{\f{1}{2} J \ba_s + \bX} - \eta \f{2\pi}{L} m_z d}} \overline{\widehat{\vp}}_{s,\eta m,m_z} \widehat{\pa{u_j}}_{\eta m,m_z}
\end{align*}
has Fourier coefficients
\begin{align*}
\widehat{\pa{F^{\eta,j,s}}}_{m} = e^{i \f{1}{2} m \ba\ind{M}^* \cdot J a_s} \sum_{m_z} e^{-i \eta \f{2\pi}{L} m_z d} \overline{\widehat{\vp}}_{s,\eta m,m_z} \widehat{\pa{u_j}}_{\eta m,m_z}
\end{align*}

On the functions given by DFTK, we remark that $\vp_s[m]$ given is periodic and that 
\begin{align*}
\cR_{\f{2\pi}{3}} \vp_{\text{Bl},s} = \tau^s \vp_{\text{Bl},s}.
\end{align*}




Finally, we analyze a symmetry. We have
\begin{align*}
	\cR_{\f{2\pi}{3}} F^{\eta,j,s} &= \int_{\R^3} \overline{\vp_{\text{Bl},s}(\by,z)} \Phi_j \pa{R_{-\f{2\pi}3} \pa{R_{\f{2\pi}{3}}\by + R_{\f{2\pi}{3}} \ba_s -2\eta J \bX},z-\eta d} \d \by \d z \\
 &= \int_{\R^3} \overline{\cR_{\f{2\pi}{3}}\vp_{\text{Bl},s}(\by,z)} \pa{\cR_{\f{2\pi}{3}} \Phi_j} \pa{\by + R_{\f{2\pi}{3}} \ba_s -2\eta J \bX,z-\eta d} \d \by \d z \\
	&= \tau^{j-s} \int_{\R^3} \overline{ \vp_{\text{Bl},s}(\by,z)} \Phi_j \pa{\by + R_{\f{2\pi}{3}} \ba_s -2\eta J \bX,z-\eta d} \d \by \d z \\
\end{align*}
and if $\vp_{\text{Bl},s}(y + R_{\f{2\pi}{3}} a_s) = \vp_{\text{Bl},s}(y + a_s)$, then 
\begin{align*}
\cR_{\f{2\pi}{3}} \pa{ \overline{F^{\eta,j,s}} F^{\eta,j',s}} = \omega^{j'-j} \; \overline{F^{\eta,j,s}} F^{\eta,j',s}
\end{align*}






\section{Effective BM coefficients from potential}%
\label{sec:bm_configuration}


From \cite{BecEmbWitZwo21}, the BM Hamiltonian is
\begin{align*}
	H = \mat{-i\sigma \na & T(x) \\ T(x)^* & -i \sigma \na},
\end{align*}
 where
\begin{align*}
	\sigma_1 = \mat{0 & 1 \\ 1 & 0} \qquad \sigma_2 = \mat{0 & -i \\ i & 0}
\end{align*}
and where
\begin{align*}
	T_1 &= \mat{w_{AA} & w_{AB} \\ w_{AB} & w_{AA}}, \\
	T_2 &= \mat{w_{AA} &  w_{AB}e^{-i\phi} \\  w_{AB}e^{i\phi} & w_{AA}}, \\
	T_3 &= \mat{w_{AA} &  w_{AB}e^{i\phi} \\  w_{AB}e^{-i\phi} & w_{AA}}
\end{align*}
and where, for $x \in \R^2$,
\begin{align*}
	T(x) := \sum_{j=1}^3 T_j e^{-iq_j \cdot x} = \mat{w_{AA} G(x) & w_{AB} \overline{F(-x)} \\ w_{AB} F(x) & w_{AA} G(x)}
\end{align*}
Now, since $q_{2,3} - q_1 = a^*_{\text{M},j}$, we know that
\begin{align*}
G(x) &= e^{-iq_1 x} \pa{1 + e^{-i a_{\text{M} ,1}^* x} +e^{-i a_{\text{M},2}^* x}} \\
F(x) &= e^{-iq_1 x} \pa{1 + \omega e^{-i a_{\text{M} ,1}^* x} + \omega^2 e^{-i a_{\text{M},2}^* x}}
\end{align*}

We have $\bbV^{1,1} \simeq w\ind{AA} G$ so $\ps{G,\bbV^{1,1}} \simeq w\ind{AA} \int_{\Omega\ind{M}} \ab{G}^2 = 3 \ab{\Omega\ind{M}} w\ind{AA}$ and hence 
\begin{align*}
	w\ind{AA} &\simeq \f{\ps{G,\bbV^{1,1}}}{3 \ab{\Omega\ind{M}}} = \f{1}{3 \sqom}\pa{\widehat{\bbV}^{1,1}_{0,0} +\widehat{\bbV}^{1,1}_{-1,0} + \widehat{\bbV}^{1,1}_{0,-1}} \\
	w\ind{AB} &\simeq \f{\ps{F,\bbV^{1,2}}}{3 \ab{\Omega\ind{M}}} = \f{1}{3 \sqom}\pa{\widehat{\bbV}^{1,2}_{0,0} + \omega\widehat{\bbV}^{1,2}_{-1,0} + \omega^2 \widehat{\bbV}^{1,2}_{0,-1}}
\end{align*}





\section{Change of gauge on the phasis of wavefunctions}%
\label{sec:change_of_gauge_on_the_phasis_of_wavefunctions}

When we change $\Phi_1 \rightarrow \Phi_1 e^{i\alpha}$, then $u_1 \rightarrow u_1 e^{i\alpha}$, $u_2 \rightarrow u_2 e^{-i\alpha}$ because $u_2(x) = \overline{u_1(-x)}$, and hence
\begin{align*}
\boxed{\overline{u_1} u_2 \rightarrow \overline{u_1} u_2 e^{-2i\alpha}}
\end{align*}
We define
\begin{align*}
	\cU := \mat{U & 0 \\ 0 & U}
\end{align*}
have
\begin{align*}
	\mat{U & 0 \\ 0 & U} \mat{\bbW^+ & \bbV \\ \bbV^* & \bbW^-}  \mat{U^* & 0 \\ 0 & U^*} = \mat{U\bbW^+ U^* & U\bbV U^*\\ U\bbV^*U^* & U\bbW^-U^*}
\end{align*}
and with $U := \mat{e^{i\alpha} & 0 \\  0 & e^{-i\alpha}}$, we have
\begin{align*}
	U \mat{B^+ & B \\ B^* & B^-} U^* = \mat{B^+ & B e^{2i\alpha} \\ B^* e^{-2i\alpha} & B^-}
\end{align*}
hence if we define $H_\alpha$ to be $H$ with $u_1 \rightarrow u_1 e^{i\alpha}$, we have that
\begin{align*}
\cU H_\alpha \cU^*
\end{align*}
is constant in $\alpha$, where $\cU := \mat{U & 0 \\ 0 & U}$.





\section{Symmetries}%
\label{sec:symmetries}


\subsection{Particle-hole}%
\label{sub:particle_hole}

We define
\begin{align*}
\cS u(x) := i \mat{0 & -\1_{2\times 2} \\ \1_{2\times 2} & 0} u(-x)
\end{align*}
We have
\begin{align*}
\cS \mat{0 & B \\ B^* & 0} \cS = -\mat{0 & B^*(-x) \\ B(-x) & 0}
\end{align*}
We have $T(-x)^* = T(x)$ hence we should have that
\begin{align*}
\cS H \cS = - H
\end{align*}


For any function $B$ and any vector function $\bm{A}$, we have
\begin{align*}
	\cS \mat{0 & B(\bX) \\ B^*(\bX) & 0} \cS &= -\mat{0 & B^*(-\bX) \\ B(-\bX) & 0} \\
	\cS \mat{0 & B(\bX) \Delta \\ B^*(\bX)\Delta & 0} \cS &= -\mat{0 & B^*(-\bX)\Delta \\ B(-\bX)\Delta & 0} \\
	\cS\mat{0 & i\bm{A}(\bX) \cdot \na \\ i\bm{A}(\bX)^* \cdot \na & 0} \cS &= \mat{0 & i \bm{A}(-\bX)^* \cdot \na \\ i\bm{A}(-\bX) \cdot \na & 0},
\end{align*}
we also compute that
\begin{align*}
	\cS \mat{\sigma \cdot \na & 0 \\ 0 & \sigma \cdot \na} \cS &= - \mat{\sigma \cdot \na & 0 \\ 0 & \sigma \cdot \na},
\end{align*}
hence if the operator $\Gamma$ is a linear combination of the terms
\begin{multline*}
\mat{\sigma \cdot \pa{-i\na} & 0 \\ 0 & \sigma \cdot \pa{-i\na}}, \mat{\sigma \cdot J\pa{-i\na} & 0 \\ 0 & \sigma \cdot J\pa{-i\na}}, \\
\mat{0 & \bbV \\ \bbV^* & 0}, \mat{0 & \Sigma \\ \Sigma^* & 0}, \mat{0 & \Sigma \Delta	 \\ \Sigma^* \Delta & 0}
\end{multline*}
it satisfies the symmetry $\cS \Gamma \cS = - \Gamma$, and those are the particle-hole symmetric terms of our effective Hamiltonian. However, if $\Gamma$ is a linear combination of the operators
\begin{align*}
	\mat{0 & \bm{\cA}\cdot\pa{-i\na} \\ \bm{\cA}^* \cdot \pa{-i\na} & 0}, \mat{0 & \bm{\cA}\cdot J\pa{-i\na} \\ \bm{\cA}^* \cdot J\pa{-i\na} & 0},  \\
	\mat{ -\f 12 \Delta & 0 \\ 0 & -\f 12 \Delta}, \mat{\bbW & 0 \\ 0 & \bbW^*}, \mat{\1_{2\times 2} & 0 \\ 0 & \1_{2\times 2}}
\end{align*}
of the effective Hamiltonian $\cH_{d,\theta}$, it satisfies $\cS \Gamma \cS = \Gamma$ and hence break the particle-hole symmetry.

But now we also compute that
\begin{multline*}
	\cS \mat{k & 0 \\ 0 & k} \cS = k, \\
	\cS \mat{\sigma \pa{-i\na +k} & 0 \\ 0 & \sigma\pa{-i\na +k}} \cS = -\mat{\sigma\pa{-i\na -k} & 0 \\ 0 & \sigma\pa{-i\na -k}}
\end{multline*}

\subsection{Mirror}%
\label{sub:mirror}

First, for any function $B$, we have $\sigma_1 B^* \sigma_1 = \mat{\overline{B_{22}} & \overline{B_{12}} \\ \overline{B_{21}} & \overline{B_{11}}}$.


The mirror operator for the BM Hamiltonian is
\begin{align*}
	\cM u(\bX) := \mat{0 & \sigma_1 \\ \sigma_1 & 0} u(\overline{\bX})
\end{align*}
where $\overline{\bX} := (X_1,-X_2) =: M \bX$, it satisfies $\cM = \cM^{-1} = \cM^*$.


Next,
\begin{align*}
	\cM \mat{0 & B(\bX) \\ B(\bX)^* & 0} \cM = \mat{0 & \sigma_1 B^*(\overline{\bX}) \sigma_1 \\ \sigma_1 B(\overline{\bX}) \sigma_1 & 0}%= \mat{0 & \sigma_1 B(\bX)^* \sigma_1 \\ \sigma_1 B(\bX) \sigma_1 & 0}
\end{align*}
and for BM's potential, $\sigma_1 T^*(M \bX) \sigma_1 = T(\bX)$




\section{Plan}%
\label{sec:plan}

Given a macroscopic model, BM of ours, we need to proceed the following way to build the band diagrams numerically
\begin{enumerate}
	\item We rescale the model and remove dimensions by applying the conjugation $\f{1}{v\ind{F} k_\theta^3} S \cdot S^*$ as in \eqref{def:conjugation}
	\item We conjugate by $U = \mat{e^{i K_1 x} & 0 \\ 0 & e^{iK_2 x}}$ to remove the $e^{-iq_1 x}$ factors
\end{enumerate}



\section{Treating the Bistritzer-MacDonal model}
\label{sec:comparision_between_bm_and_our_model}

In this section, we apply the plan of Section \ref{sec:plan} to treat the BM model.

\subsection{Rescaling}%
\label{sub:rescaling}

We consider 
\begin{align*}
T(x) = \sum_{j=1}^3 T_j e^{-i q_j x}, \qquad q_{2,3} = \mat{\pm \sqrt{3}/2 \\ 1/2}, \qquad q_1 = - q_2 - q_3.
\end{align*}
The BM Hamiltonian is
\begin{align*}
\mat{-i v\ind{F} \sigma \na & w T(k_\theta x) \\ w T^*(k_\theta x) & -iv\ind{F} \sigma \na}.
\end{align*}
We consider the rescaling
\begin{align*}
Su(x) := u\pa{\f{x}{k_\theta}}, \qquad S^*u(y) = k_\theta^2 u\pa{k_\theta y}, \qquad S S^* = k_\theta^2
\end{align*}
where we defined $S^*$ as $\int_\Omega \overline{f} \; Sg = \int_{L\Omega/k_\theta} g \; \overline{S^*f}$.
We have $\na S^* = k_\theta S^* \na$ so $S \na S^* = k_\theta^3 \na$ and $SfS^* = k_\theta^2 f\pa{\f{x}{k_\theta}}$ so when $x = y k_\theta$ is the microscopic scale
\begin{align}\label{def:conjugation}
	\f{1}{k_\theta^3 v\ind{F}} S \pa{\mat{-i v\ind{F} \sigma \na & w T(k_\theta x) \\ w T^*(k_\theta x) & -iv\ind{F} \sigma \na} -E} S^* = \mat{-i\sigma \na & \alpha T\pa{x} \\ \alpha T^*(x) & -i\sigma \na} - \ep =: H_{BM}
\end{align}
where $\alpha := \f{w}{k_\theta v\ind{F}}$ and where $\ep = \f{E}{v\ind{F} k_\theta}$ is the unit of \cite[Fig 1]{TarKruVis19} defined in the caption.

\subsection{Removing $e^{-iq_1 x}$}%


With 
\begin{align}\label{def:U}
U := \mat{e^{i K_1 x} \1_2 & 0 \\ 0 & e^{iK_2 x} \1_2 },
\end{align}
we have
\begin{align*}
U H_{BM} U^* &= \mat{\sigma \cdot \pa{-i\na  -  K_1} & T(x) e^{i\pa{ K_1 - K_2}x} \\ T(x)^*e^{i\pa{K_2 -  K_1}x} & \sigma\cdot \pa{-i\na  -K_2}} \\
		&= \mat{\sigma \cdot \pa{-i\na - K_1} & \bf{T} \\ \bf{T}^* & \sigma \cdot \pa{-i\na -K_2}}
\end{align*}
where $\bf{T}$ is moiré-periodic.



\section{Treating our model}

\subsection{Goal}%
\label{sub:Goal}

Our Hamiltonian is
\begin{equation} \label{eq:def:Sigma_d}
    \cH = \ept^{-1} \cV + c_\theta T + \ept T^{(1)},\qquad \cH \p = \frac{E}{\ept} \cS \p
\end{equation}
where the three operators $\cV$, $T$, and $T^{(1)}$ are
\begin{align*}
\cS =  \mat{
        \bbI_2 & \Sigma \\ \Sigma^* & \bbI_2
} \qquad \mbox{and} \qquad \cV = \mat{
        \bbW^+ & \bbV \\
        \bbV^* & \bbW^-
} ,
\end{align*}
\[
    T = \mat{ v_F \bsigma \cdot ( - \ri \nabla) &  J ( - \ri \nabla \Sigma) \cdot ( - \ri \nabla) \\
         J ( - \ri \nabla \Sigma^*) \cdot ( - \ri \nabla) & v_F \bsigma \cdot ( - \ri \nabla) },
\]
\begin{equation} \label{eq:op_Td1}
    T^{(1)} = - \frac12 {\rm div} \cS \nabla \bullet + \frac12 \mat{
        - v_F \bsigma \cdot J (- \ri \nabla) & 0 \\
        0 & v_F \bsigma \cdot J (- \ri \nabla) }.
\end{equation}
and with $A = -i\na \Sigma$.

Numerically, we will discretize $\cH$ and $\cS$ and compute the eigenvalues of $\cS^{-\f 12}\cH\cS^{-\f 12}$.

\subsection{Gauge change}%
\label{sub:Gauge change}


We recall that
\begin{align*}
U := \mat{e^{i K_1 x} \1_2 & 0 \\ 0 & e^{iK_2 x} \1_2 },
\end{align*}
and that we work on $U \cH U^*$

\subsubsection{Electric potentials and mass matrix}%

We have
\begin{align*}
U \mat{0 & \bbV \\ \bbV^* & 0} U^* &= \mat{0 & \widetilde{\bbV} \\ \widetilde{\bbV}^* & 0},\qquad U \mat{\bbW^+ & 0 \\ 0 & \bbW^-} U^* &= \mat{\bbW^+ & 0 \\ 0 & \bbW^-}
\end{align*}
and
\begin{align*}
U \mat{0 & \Sigma \\ \Sigma^* & 0} U^* &= \mat{0 & \widetilde{\Sigma} \\ \widetilde{\Sigma}^* & 0}
\end{align*}

\subsubsection{First order differential operators}%
\label{ssub:First order differential operators}

We compute
\begin{align*}
	U \mat{\sigma \pa{-i\na} & 0 \\ 0 & \sigma \pa{-i\na}} U^* = \mat{\sigma \pa{-i\na -  K_1} & 0 \\ 0 & \sigma \pa{-i\na- K_2}} 
\end{align*}
and as presented in Section \ref{sec:Appendix : more on the magnetic term}, with $A := -i\na\Sigma$,
\begin{multline*}
	U \mat{0 & JA (-i\na) \\ \pa{JA}^* (-i\na)} U^* \\
=	\mat{0 & J \widetilde{A} \cdot(-i\na - K_2) \\  \pa{J\widetilde{A}}^* \cdot(-i\na -  K_1) & 0}
\end{multline*}
Moreover,
\begin{multline*}
U  \mat{- v_F \sigma\cdot J(-i\na) & 0 \\ 0 & v_F \sigma\cdot J(-i\na)} U^*  \\
=  \mat{- v_F \sigma\cdot J(-i\na -  K_1) & 0 \\ 0 & v_F \sigma\cdot J(-i\na - K_2)}
\end{multline*}


\subsubsection{Second order differential operator}

We have
\begin{align*}
-i \div e^{i K_1 x} \circ = -i\div +  K_1, \qquad -i\na e^{-iK_2 x} \circ = -i\na - K_2
\end{align*}
so
\begin{align*}
e^{i K_1 x} (-i \div) \Sigma (-i\na) e^{-iK_2 x} &= \pa{-i \div -  K_1} e^{i ( K_1 - K_2) x} \Sigma \pa{-i\na - K_2} \\
&= \pa{-i \div -  K_1} \widetilde{\Sigma} \pa{-i\na - K_2}
\end{align*}
and as operator compositions,
\begin{multline*}
U(-i\div ) \cS (-i\na) U^* = U \mat{-\Delta & (-i \div) \Sigma (-i\na) \\ (-i \div) \Sigma^* (-i\na) & -\Delta} U^* \\
= \mat{ \pa{-i\na - K_1}^2 &\pa{-i \div -  K_1} \widetilde{\Sigma} \pa{-i\na - K_2} \\ \pa{-i \div - K_2} \widetilde{\Sigma}^* \pa{-i\na -  K_1} & \pa{-i\na - K_2}^2}
\end{multline*}


\section{Discretization}%
\label{sec:Discretization}


Now we take the basis
\begin{align*}
e_m^a := e^a \otimes \f{e^{i m a\ind{M}^* x}}{\sqom}
\end{align*}
 where
\begin{align*}
e^1 := \mat{1 \\ 0 \\ 0 \\ 0},\dots
\end{align*}




\subsection{Electric potentials}%
\label{sub:Electric potentials}


With $\bbV = \sum_m V_m e_m$, $e_m := \f{e^{im a\ind{M}^*x}}{\sqom}$, we have
\begin{align*}
\ps{e_n,\bbV e_p} = \f{V_{n-p}}{\sqom}
\end{align*}
so for multiplication operatos $A,B,C,D$,
\begin{align*}
	\ps{e_n, \mat{A & B \\ C & D} e_p} = \f 1\sqom \mat{ A_{n-p} & B_{n-p} \\ C_{n-p} & D_{n-p}}
\end{align*}


\subsection{First order differential operators}%
\label{sub:First order differential operators}
We have
\begin{align*}
\ps{e_n, \sigma \cdot \pa{-i\na} e_p} = \delta_{n-p} \sigma \cdot n a\ind{M}^*
\end{align*}
and with $\bbA = \sum_m A_m e_m$,
\begin{align*}
\ps{e_n,\bbA \cdot (-i\na) e_p} = \f{A_{n-p}}{\sqom} \cdot p  a\ind{M}^*
\end{align*}
hence
\begin{align*}
\ps{e_n,U \mat{\sigma \pa{-i\na} & 0 \\ 0 & \sigma \pa{-i\na}} U^* e_p} = \delta_{n-p} \mat{\sigma \pa{n a\ind{M}^* -  K_1} & 0 \\ 0 & \sigma \pa{n a\ind{M}^*- K_2}} 
\end{align*}
and
\begin{multline*}
	\ps{e_n,U \mat{0 & JA (-i\na) \\ \pa{JA}^* (-i\na)} U^* e_p} \\
= \f{1}{\sqom}\mat{0 & \pa{J \widetilde{A}}_{n-p} \cdot(p  a\ind{M}^* - K_2) \\  \pa{J \widetilde{A}}^*_{n-p} \cdot(p a\ind{M}^* -  K_1) & 0}
\end{multline*}
Moreover,
\begin{multline*}
	\ps{e_n,U  \mat{- v_F \sigma\cdot J(-i\na) & 0 \\ 0 & v_F \sigma\cdot J(-i\na)} U^* e_p}  \\
=  \delta_{n-p}\mat{- v_F \sigma\cdot J(p a\ind{M}^* -  K_1) & 0 \\ 0 & v_F \sigma\cdot J(p a\ind{M}^* - K_2)}
\end{multline*}


\subsection{Second order differential operator}%
\label{sub:Second order differential operator}

We have
\begin{align*}
\ps{e_p, -i\div \pa{\Sigma \pa{-i\na} e_m}} = \ps{\pa{-i\na} e_p,\Sigma \pa{-i\na} e_m} = p a\ind{M}^* \cdot m a\ind{M}^* \f{\Sigma_{p-m}}{\sqom}
\end{align*}
so
\begin{multline*}
\f 12 \ps{e_n, U(-i\div ) \cS (-i\na) U^*e_p} = \f 12 U \mat{-\Delta & (-i \div) \Sigma (-i\na) \\ (-i \div) \Sigma^* (-i\na) & -\Delta} U^*  \\
= \f{1}{2} \mat{\pa{p a\ind{M}^* - K_1}^2 &\pa{p a\ind{M}^* -  K_1} \cdot \pa{n a\ind{M}^* - K_2} \f{\widetilde{\Sigma}_{n-p}}{\sqom}  \\ \pa{p a\ind{M}^* - K_2}\cdot\pa{n a\ind{M}^* -  K_1} \f{\widetilde{\Sigma}^*_{n-p}}{\sqom}  & \pa{p a\ind{M}^* - K_2}^2}
\end{multline*}




\section{Appendix : more on the magnetic term}%
\label{sec:Appendix : more on the magnetic term}

We have
\begin{align*}
A := -i\na \Sigma = e^{-iq_1 x} \pa{-i\na - q_1} \db{u_j,u_{j'}}^{+-}
\end{align*}
hence with $\widetilde{f} := e^{iq_1 x} f$,
\begin{align*}
\widetilde{A} = \pa{-i\na - q_1} \widetilde{\Sigma}
\end{align*}

\begin{multline*}
	U \mat{0 & JA (-i\na) \\ JA^* (-i\na)} U^* \\
	= \mat{0 & e^{i( K_1 - K_2)x} JA \cdot(-i\na - K_2) \\ e^{i(K_2 -  K_1)x} JA^* \cdot(-i\na -  K_1) & 0} \\
	= \mat{0 & e^{iq_1 x} JA \cdot(-i\na - K_2) \\ e^{-iq_1 x} JA^* \cdot(-i\na -  K_1) & 0} \\
	= \mat{0 & J \widetilde{A} \cdot(-i\na - K_2) \\  J\widetilde{A}^* \cdot(-i\na -  K_1) & 0}
\end{multline*}
Now
\begin{align*}
\div JA = 0,\qquad -i\div J \widetilde{A} = q_1 J \widetilde{A}
\end{align*}
and since $\widetilde{A}^* = e^{-iq_1 x} A^*$, then $-i\div J \pa{\widetilde{A}^*} = -q_1 J  \pa{\widetilde{A}^*}$. We have
\begin{align*}
\div A = \sum_m \pa{A_m^1 \pa{m  a\ind{M}^*}_1 + A_m^2 \pa{m  a\ind{M}^*}_2} \f{e^{im a\ind{M}^*x}}{\sqom}
\end{align*}

With $A$ a $4 \times 4$ matrix, computing $\ps{v, A u} = \ps{\mat{v_1 \\ v_2},\mat{A_{11} & A_{12} \\ A_{21} & A_{22}} \mat{u_1\\u_2}}$, we compute that $A^{*_m}$, the pointwise dual of $A$ at each $x$, is indeed the hermitian conjugate for any $x$. More precisely,
\begin{align*}
\ps{u,v} = \int_X \ps{u,v}_{mat}
\end{align*}
so
\begin{align*}
\ps{u,V v} = \int_X \ps{u(x),V(x) v(x)}_{mat} = \int_X \ps{V(x)^{*_m} u(x),v(x)}_{mat} = \ps{V^{*_m} u,v}
\end{align*}
and
\begin{align*}
\boxed{V^* = V^{*_m}}
\end{align*}


We remark also that $J A J = - \pa{A^{-1}}^T$
The action of $J$ is on the composants of $A$, not on $u$ !!! So we have 
\begin{align*}
A = \mat{A^{(1)} \\ A^{(2)}} =: \mat{C \\ B} = \mat{\mat{C_{11} & C_{12} \\ C_{21} & C_{22}} \\ \mat{B_{11} & B_{12} \\ B_{21} & B_{22}}}
\end{align*}
and $A$ acts on $u$ as $A u = \mat{A^{(1)} u \\ A^{(2)} u}$ hence $A^* = \mat{\pa{A^{(1)}}^* \\ \pa{A^{(2)}}^*}$ and
\begin{align*}
JA = \mat{-A^{(2)} \\ A^{(1)}}, \qquad \pa{JA}^* = \mat{-\pa{A^{(2)}}^* \\ \pa{A^{(1)}}^*} = J A^* \neq -A^* J !!
\end{align*}
We recall that $\partial_j$ acts on $L^2(\R^d,\C^2)$ as
\begin{align*}
-i \partial_j u = \mat{-i\partial_j u_1 \\ -i\partial_j u_2}, \qquad -i\na u = \mat{-i\partial_1 u \\ -i\partial_2 u} = \mat{\mat{-i\partial_1 u_1 \\ -i\partial_1 u_2} \\ \mat{-i\partial_2 u_1 \\ -i\partial_2 u_2}}
\end{align*}
so $\pa{-i\partial_j}^* = -i\partial_j$ and $\pa{-i\na}^* = -i\na$. For any $4\times 4$ valued function $B$, wwe have
\begin{align*}
\partial_j \pa{B u} = \partial_j \mat{B_{11} u_1 + B_{12} u_2 \\ B_{21} u_1 + B_{22} u_2} = B \partial_j u + \pa{\partial_j B}u 
\end{align*}
where
\begin{align*}
	\partial_j B := \mat{\partial_j B_{11} & \partial_j B_{12} \\ \partial_j B_{21} & \partial_j B_{22}}
\end{align*}
i.e $\partial_j$ acts pointwise on vectors and matrices. Moreover, for $A = \mat{A^{(1)} \\ A^{(2)}}$, we have
\begin{align*}
	\div A u & = \partial_1 \pa{A^{(1)} u} + \partial_2 \pa{A^{(2)} u} = \sum_j A^{(j)} \partial_j u + \pa{\partial_j A^{(j)}}u  \\
& = \pa{\div A}u  + A \cdot \na u
\end{align*}
where we also define $\div$ acting pointwise on the $4\times 4$ matrices, i.e
\begin{align*}
\div A := \pa{\div A_{ij}}_{1 \le i,j \le 2} = \pa{\partial_1 A^{(1)}_{ij} + \partial_2 A^{(2)}_{ij}}_{ij}
\end{align*}
In this case, $\div J \na f = 0$ for any $4 \times 4$ matrix valued function $f$. Moreover,
\begin{align*}
	& \ps{V, -i\na u} = \ps{\mat{V_1 \\ V_2}, -i\na u} = \ps{\mat{V_1 \\ V_2}, \mat{-i\partial_1 u \\ -i\partial_2 u}} =\sum_j  \ps{V_j,-i\partial_j u} \\
&\qquad = \sum_j \ps{-i\partial_j V_j,u} = \ps{-i\div V, u}
\end{align*}

Hence for $A = -i\na \Sigma$,
\begin{align*}
	&\ps{v, JA \cdot \pa{-i\na -K_2} u} = \ps{\pa{JA}^* v, \pa{-i\na -K_2}u} = \ps{JA^* v, \pa{-i\na -K_2}u} \\
	& \qquad =  \ps{\pa{-i\div -K_2}JA^* v, u} \\
	& \qquad = \ps{ \pa{\pa{-i\div }\pa{JA^*}}v , u} + \ps{(JA^*) \cdot \pa{-i\na -K_2}v , u} \\
	& \qquad = \ps{(JA^*) \cdot\pa{-i\na -K_2}v , u}
\end{align*}
Repeating the same computations, we find that
\begin{align*}
& \ps{v, \pa{J \widetilde{A}} \cdot \pa{-i\na -K_2} u} \\
& \qquad = \ps{\pa{J \widetilde{A}^*} \cdot \pa{-i\na -K_2} v,u} + \ps{-i\div \pa{J \widetilde{A}^*} v,u} \\
& \qquad = \ps{\pa{J \widetilde{A}^*} \cdot \pa{-i\na - K_1} v,u}
\end{align*}
so 
\begin{align*}
\pa{\pa{J \widetilde{A}} \cdot \pa{-i\na -K_2}}^* = \pa{J \widetilde{A}^*} \cdot \pa{-i\na - K_1} = \pa{J \widetilde{A}}^* \cdot \pa{-i\na - K_1}
\end{align*}

We can compute (double checked) that for a 1-component potential $V$, $\ps{u, V v} = \ps{\overline{V} u,v}$ hence $V^{*_f} = \overline{V}$ and $\pa{V^{*_f}}_m = \overline{V}_{-m}$

Next, 
\begin{align*}
	\ps{u, A\cdot (-i\na) v} &= \int \overline{\overline{A}u} \cdot (-i\na)v = \int v \overline{(-i\na) \overline{A} u} \\
&= \ps{\pa{-i \div \overline{A}} u,v} + \ps{\overline{A}\cdot (-i\na) u,v}
\end{align*}
hence
\begin{align*}
\pa{A\cdot (-i\na)}^{*_f} = -i \div \overline{A} + \overline{A} \cdot(-i\na)
\end{align*}
and $-i \div A = q_1 \cdot A$ implies $-i\div \overline{A} = -q_1 \cdot \overline{A}$.
Now for a $4 \times 4$ matrix function $V$, we have
\begin{align*}
	\ps{e_i \otimes e_I, V^{*_f *_m} e_j \otimes e_J} &= \pa{V^{*_f *_m}}^{IJ}_{i-j} = \ps{V e_i \otimes e_I, e_j \otimes e_J} \\
&= \overline{\ps{e_j \otimes e_J, V e_i \otimes e_I}} = \overline{V^{JI}_{j-i}}
\end{align*}
hence
\begin{align*}
\pa{V^{*_f *_m}}^{IJ}_{m} = \overline{V^{JI}_{-m}}, \qquad V^{*_f *_m} = V^*
\end{align*}



\subsection{Stuff}%
\label{sub:Stuff}
Let us consider an operator $H$ and its discretization
\begin{align*}
H_{ij}^{ab} := \ps{e_i^a, H e^b_j}\qquad \pa{H^*}^{ab}_{ij} = \overline{H^{ba}_{ji}}
\end{align*}
and we want to build the Hermitian operator
\begin{align*}
	\mat{0 & H \\ H^* & 0}
\end{align*}
\begin{align*}
\pa{H^*}^t = \overline{H}, \qquad \pa{\pa{H^*}^t}^{ab}_{ij} = \overline{H^{ab}_{ij}}
\end{align*}


For $H = V$, we have
\begin{align*}
H_{ij}^{ab} := V^{ab}_{i-j} \qquad \pa{H^*}^{ab}_{ij} = \overline{V^{ba}_{j-i}},\qquad \overline{H}^{ab}_{ij} = \overline{V^{ab}_{i-j}}
\end{align*}
and in the code we implement $\overline{H^{ab}_{ij}}$.



Let us assume that $-i\div A = q_1 \cdot A$, so in Fourier space this is written
\begin{align}\label{eq:div_Fourier}
	m a\ind{M}^* \cdot A_m = q_1 \cdot A_m
\end{align}
but be careful, this relation is true only for $m \in \{-N/2,\dots,N/2\}$ !!!! Now $e_m^a := e^a \otimes \f{e^{i m a\ind{M}^* x}}{\sqom}$,
\begin{align*}
H_{ij}^{ab} := \ps{e_i^a, A \cdot \pa{-i\na -  K_1} e^b_j} = A_{i-j}^{ab} \cdot (j a\ind{M}^* - K_1)
\end{align*}
where
\begin{align*}
	\pa{H^*}^{ab}_{ij} &= \overline{H^{ba}_{ji}} = \overline{\ps{e^b_j, A\pa{-i\na - K_1} e^a_i}} =\ps{e_i^a, A^* \cdot \pa{-i\na -K_2}e^b_j} \\
&= \overline{A^{ba}_{j-i}}(j a\ind{M}^* -K_2) \underset{\substack{\refeq{eq:div_Fourier}}}{=} \; \overline{A^{ba}_{j-i}} \cdot \pa{i a\ind{M}^* - K_1} = \overline{H^{ba}_{ji}}
\end{align*}
and
\begin{align*}
\overline{H}^{ab}_{ij} = \overline{A_{i-j}^{ab}} \cdot (j a\ind{M}^* - K_1) = \overline{A_{i-j}^{ab}} \cdot (i a\ind{M}^* -K_2)
\end{align*}
and the last inequality is true only when $i-j \in \{-N/2,\dots,N/2\}$ !!!






\bibliographystyle{siam}
\bibliography{/home/louis/Documents/biblio.bib}
\end{document}
