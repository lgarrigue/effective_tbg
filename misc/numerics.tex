\documentclass[11pt,a4paper,reqno,french,tikz]{amsart}
\usepackage[utf8]{inputenc} \usepackage[T1]{fontenc}
%\usepackage{babel} % français \usepackage[latin1]{inputenc}
\usepackage{amsthm,amsmath,amsfonts,amssymb,amsxtra,appendix,bookmark,dsfont,bm,mathrsfs,amstext,amsopn,mathrsfs,mathtools,comment,cite,hyperref,color,xcolor,cite}

\usepackage{stmaryrd}

%% Julia definition for Listings package (c) 2014 Jubobs
\usepackage{beramono,listings}
\lstdefinelanguage{Julia}%
  {morekeywords={abstract,break,case,catch,const,continue,do,else,elseif,end,export,false,for,function,immutable,import,importall,if,in,macro,module,otherwise,quote,return,switch,true,try,type,typealias,using,while},sensitive=true,alsoother={$},morecomment=[l]\#,morecomment=[n]{\#=}{=\#},morestring=[s]{"}{"},morestring=[m]{'}{'},}[keywords,comments,strings]
\lstset{language= Julia,basicstyle= \ttfamily,keywordstyle= \bfseries\color{blue},stringstyle= \color{magenta},commentstyle = \color{ForestGreen},showstringspaces = false,}
%% END Julia

%%%%%%%%%%%%%%%%%%%% Standard macros
% Environments
\newtheorem{theorem}{Theorem}[section]\newtheorem{definition}[theorem]{Definition}\newtheorem{lemma}[theorem]{Lemma}\newtheorem{example}[theorem]{Example}\newtheorem{proposition}[theorem]{Proposition}\newtheorem{corollary}[theorem]{Corollary}\newtheorem{conjecture}[theorem]{Conjecture}\newtheorem{remark}[theorem]{Remark}

% Mathbb
\let\C\relax\newcommand{\C}{\mathbb{C}}\newcommand{\Z}{\mathbb{Z}}\newcommand{\R}{\mathbb{R}}\newcommand{\N}{\mathbb{N}}\newcommand{\Q}{\mathbb{Q}}\newcommand{\bbM}{\mathbb{M}}

% Mathcal
\newcommand\cA{\mathcal{A}}\newcommand\cB{\mathcal{B}}\newcommand\cC{\mathcal{C}}\newcommand\cD{\mathcal{D}}\newcommand\cE{\mathcal{E}}\newcommand\cF{\mathcal{F}}\newcommand\cG{\mathcal{G}}\newcommand\cH{\mathcal{H}}\newcommand\cI{\mathcal{I}}\newcommand\cJ{\mathcal{J}}\newcommand\cK{\mathcal{K}}\newcommand\cL{\mathcal{L}}\newcommand\cM{\mathcal{M}}\newcommand\cN{\mathcal{N}}\newcommand\cO{\mathcal{O}}\newcommand\cP{\mathcal{P}}\newcommand\cQ{\mathcal{Q}}\newcommand\cR{\mathcal{R}}\newcommand\cS{\mathcal{S}}\newcommand\cT{\mathcal{T}}\newcommand\cU{\mathcal{U}}\newcommand\cV{\mathcal{V}}\newcommand\cW{\mathcal{W}}\newcommand\cX{\mathcal{X}}\newcommand\cY{\mathcal{Y}}\newcommand\cZ{\mathcal{Z}}

% Math operators
\DeclareMathOperator{\tr}{Tr}\DeclareMathOperator{\ran}{Ran}\DeclareMathOperator{\Span}{Span}\DeclareMathOperator{\Ker}{Ker}\DeclareMathOperator{\re}{Re}\DeclareMathOperator{\id}{id}\DeclareMathOperator{\im}{Im}\DeclareMathOperator{\dist}{dist}\let\div\relax\DeclareMathOperator{\div}{div}
\def\d{{\rm d}}

% Others
\renewcommand{\ge}{\geqslant}\renewcommand{\le}{\leqslant}

\newcommand{\intent}[1]{\llbracket #1 \rrbracket}

%%%%%%%%%%%%%%%%%%%% Macros Louis
% Delimiters
\newcommand{\pa}[1]{\left( #1 \right)} % ()
\newcommand{\acs}[1]{\left\{ #1 \right\}} % {}
\newcommand{\seg}[1]{\left[ #1 \right]} % []
\newcommand{\ab}[1]{\left|#1\right|} % ||
\newcommand{\ps}[1]{\left< #1 \right>} % <>
\newcommand{\proj}[1]{\big| #1 \big> \big< #1 \big|} % |u><u|
\newcommand{\nor}[2]{ \left| \! \left| #1 \right| \! \right|_{#2} } % ||.||

% Greek letters shortcuts
\newcommand\vp{\varphi} % φ
\def\eps{\varepsilon}\newcommand{\ep}{\varepsilon} % ε
\let\p\relax\newcommand{\p}{\psi} % ψ
\newcommand{\na}{\nabla} % ∇

% Others
\newcommand{\f}[2]{\frac{#1}{#2}} % fraction
\newcommand{\bul}{$\bullet$ \hspace{0.1cm}} % •
\newcommand{\mymax}[1]{\underset{\substack{#1}}{\text{\normalfont{max}}}\quad} % max
\newcommand{\mymin}[1]{\underset{\substack{#1}}{\text{\normalfont{min}}}\quad} % min
\newcommand{\mysup}[1]{\underset{\substack{#1}}{\text{\normalfont{sup}}}\quad} % sup
\newcommand{\myinf}[1]{\underset{\substack{#1}}{\text{\normalfont{inf}}}\quad} % inf
\newcommand{\ind}[1]{_{\textup{#1}}} % indice
\newcommand{\apo}[1]{#1''} % apostrophes "..."
\newcommand{\mat}[1]{\begin{pmatrix} #1 \end{pmatrix}} % matrices
\def\1{{\mathds{1}}}
\newcommand{\ketbra}[2]{\left| #1 \right> \left< #2 \right|}
\newcommand{\bbV}{\mathbb{V}}


\newcommand{\db}[1]{\left(\!\left( #1 \right)\!\right)}
\def\bX{{\mathbf X}}
\def\bG{{\mathbf G}}
\def\ba{{\mathbf a}}
\def\bb{{\mathbf b}}
\def\be{{\mathbf e}}
\def\bk{{\mathbf k}}
\def\bp{{\mathbf p}}
\def\bq{{\mathbf q}}
\def\br{{\mathbf r}}
\def\bx{{\mathbf x}}
\def\bmm{{\mathbf m}}
\def\bv{{\mathbf v}}
\def\bs{{\mathbf s}}
\def\by{{\mathbf y}}
\def\bn{{\mathbf n}}
\def\bA{{\mathbf A}}
\def\bH{{\mathbf H}}
\def\bK{{\mathbf K}}
\def\bR{{\mathbf R}}
\def\bS{{\mathbf S}}
\def\gR{{\mathfrak R}}

\def\bbI{{\mathbb I}}
\def\bbV{{\mathbb V}}
\def\bbW{{\mathbb W}}
\def\bbA{{\mathbb A}}
\def\bbB{{\mathbb B}}
\def\1{{\mathds{1}}}
\def\L{{\mathbb L}}

\newcommand{\dd}{\tfrac{d}{2}}
\newcommand{\sqom}{\sqrt{\ab{\Omega\ind{M}}}}

\def\lAngle{\langle\!\langle}
\def\rAngle{\rangle\!\rangle}

%%%%%%%%%%%%%%%%%%%% End macros

\title[Numerics effective TBG]{Numerical computations for an effective model of twisted bilayer graphene}
\author[É. Cancès, L. Garrigue and D. Gontier]{Éric Cancès, Louis Garrigue and David Gontier}
% \address{CERMICS, \'Ecole des ponts ParisTech, 6 and 8 av. Pascal, 77455 Marne-la-Vallée, France} 
% \email{louis.garrigue@enpc.fr}
% \date{\today}
\begin{document}
\maketitle

\section{Standard monolayer}%
\label{sec:standard_monolayer}

We choose for the microscopic lattice, the orientation
\begin{multline}\label{eq:lattice_a}
a_1 = a \mat{ \f{\sqrt 3}2 \\ \f 12}, \qquad a_2 = a \mat{ \f{\sqrt 3}2 \\ -\f 12} \\
a^*_1 = \f{2\pi}{a} \mat{ \f{1}{\sqrt 3} \\ 1}= \f{4\pi}{a\sqrt{3}} \mat{ \f{1}{2} \\ \f{\sqrt{3}}{2}}, \qquad a^*_2 = \f{2\pi}{a} \mat{ \f{1}{\sqrt 3} \\ -1}= \f{4\pi}{a\sqrt{3}} \mat{ \f{1}{2} \\ -\f{\sqrt{3}}{2}}
\end{multline}
and for the Macroscopic lattice, we choose the orientation
\begin{multline}\label{def:lattice_b}
b_1 = b \mat{ -\f 12 \\ \f{\sqrt 3}{2}  }, \qquad b_2 = b \mat{ \f 12 \\ \f{\sqrt 3}2}, \quad b^*_1 = \f{4\pi}{b\sqrt{3}} \mat{- \f{\sqrt{3}}{2}\\ \f{1}{2} }, \qquad b^*_2 = \f{4\pi}{b\sqrt{3}} \mat{\f{\sqrt{3}}{2} \\ \f{1}{2} }
\end{multline}
so $-J b_j^* = \f{a}{b} a_j^*$ and $J b_j = \f{b}{a} a_j$ and 
\begin{align*}
	\cM_b := \mat{b_1 & b_2} = \f{b}{2} \mat{-1 & 1 \\ \sqrt 3 & \sqrt 3}
\end{align*}

In reduced coordinates, with 
\begin{multline*}
	\cM : \mathbb{T}^2 \simeq [0,1]^2 \rightarrow \Omega, \\
	\cM :=  \f a2 \mat{\sqrt{3} & \sqrt{3} \\ 1 & -1} = \mat{a_1 & a_2}, \qquad  \cM^{-1} = \f 1a \mat{\f 1{\sqrt 3} & 1 \\ \f 1{\sqrt 3} & -1}
\end{multline*}
and
\begin{align*}
	2\pi \pa{\cM^{-1}}^* = \mat{a_1^* & a_2^*} = \f{2\pi}a \mat{\f{1}{\sqrt 3} & \f{1}{\sqrt 3} \\ 1 & -1} =: S
\end{align*}

\subsection{Dirac point}%
\label{sub:dirac_point}

We have
\begin{align*}
K = \f{-a_1^* +a_2^*}{3}, \qquad a_1^*\cdot a_2^* = -\f{|a_j^*|^2}{2}, \qquad \ab{K} = \f{|a_j^*|}{\sqrt 3}
\end{align*}

\subsection{From $q$ to $m_q$}%
\label{sub:from_q_to_m_q_}

Suppose you know $q$ in cartesian coordinates and you want to compute $m^q$, its reduced coordinates, that is $m^q a = q$, then since $m^q a = \pa{a_1^* a_2^*} \mat{m^q_1 \\ m^q_2} = 2\pi \pa{\cM^{-1}}^* \mat{m^q_1 \\ m^q_2}$,
\begin{align}\label{eq:inv_q}
\mat{m^q_1 \\ m^q_2} = \f{1}{2\pi} \cM^* q
\end{align}

\subsection{Fourier conventions}%
\label{sub:fourier_conventions}

We will manipulate functions which are $\Omega$-periodic in $\bx$, but not in $z$, our Fourier transform conventions will be
\begin{align*}
	\pa{\cF f}_m(k_z) := \f{1}{2\pi \ab{\Omega}} \int_{\Omega\times\R} e^{-i\pa{ma^* \bx + k_z z}} f(\bx,z) \d \bx \d z
\end{align*}
hence any function can be decomposed as
\begin{align*}
f(\bx,z) = \sum_{m \in \Z^d} \int_\R e^{i\pa{ma^* \bx + k_z z}} f_\bG(k_z) \d k_z
\end{align*}
We also recall that $\int_\R e^{ipz} \d z = 2\pi \delta(p)$.


Now we consider that $f$ and $g$ are $L$-periodic in $z$, and $\int_\R \d z \simeq \int_{[0,L]} \d z$ so the Fourier transform is
\begin{align*}
	\boxed{\pa{\cF f}_{m,m_z} := \f{1}{\Gamma} \int_{\Omega\times [0,L]} e^{-i\pa{ma^* \bx + m_z \f{2\pi}{L} z}} f(\bx,z) \d \bx \d z}
\end{align*}
where $\Gamma := \sqrt{L\ab{\Omega}}$
and the reconstruction formula is
\begin{align}\label{eq:dec_four}
\boxed{f(\bx,z) =  \sum_{\substack{\bmm \in \Z^2 \\ m_z \in \Z}}  \f{e^{i\pa{\bmm \ba^* \cdot \bx + m_z \f{2\pi}L z}}}{\Gamma} \widehat{f}_{\bmm,m_z}}
\end{align}
We define the scalar product
\begin{align*}
\ps{f,g} := \int_{\Omega\times [0,L]} \overline{f}g
\end{align*}
and compute Plancherel's formula
\begin{align}\label{eq:plancherel}
\ps{f,g} = \sum_{\substack{\bmm \in \Z^2 \\ m_z \in \Z}} \overline{\widehat{f}_{\bmm,m_z}} \widehat{g}_{\bmm,m_z}.
\end{align}
Hence, as a verification, we test that the normalization of the $\widehat{u_j}$'s is the right one by checking that $\nor{u_j}{L^2}^2 = 1$ via \eqref{eq:plancherel}.

We implement the Fourier transforms
\begin{lstlisting}
myfft(a,B) = fft(a)*sqrt(B)/length(a)
myifft(a) = ifft(a)*length(a)/sqrt(B)
\end{lstlisting}
where $B = \Gamma^2 = L \ab{\Omega}$ in $3d$, $B = L$ in $1d$ in $z$, and $B=\ab{\Omega}$ in $2d$ in $(x,y)$. If $a_i = f(x_i)$ are the actual values of the functions, then $myfft(a)[m] \simeq \pa{\cF f}_{m-1}$ up to Riemann series errors.


\subsection{Rotation action}%
\label{sub:rotation_action}


We know that $R_{\f{2\pi}3} \pa{ma^* } = \pa{R_{\f{2\pi}3}^{\text{red}} m} a^*$ where
\begin{align*}
	R_{\f{2\pi}3}^{\text{red}} = S^{-1} R_{\f{2\pi}3} S =  \cM^* R_{\f{2\pi}3} \pa{\cM^*}^{-1} = \mat{-1 & 1 \\ -1 & 0}, \qquad R_{-\f{2\pi}3}^{\text{red}} = \mat{0 & -1 \\ 1 & -1}
\end{align*}
and
\begin{align*}
\cR_{\f{2\pi}3} f(x) = \sum_m f_m e^{i \pa{R_{\f{2\pi}3}^{\text{red}} m} a^* \cdot x} = \sum_m f_{R_{-\f{2\pi}3}^{\text{red}} m} e^{im a^* \cdot x}
\end{align*}

Similarly, $R_{\f{\pi}2}  \pa{ma^* } = \pa{R_{\f{\pi}2}^{\text{red}} m} a^*$ where
\begin{align*}
	R_{\f{\pi}2} ^{\text{red}} = S^{-1} R_{\f{\pi}2}  S = \f{1}{\sqrt 3} \mat{-1 & 2 \\ -2 & 1}, \qquad R_{-\f{\pi}2}^{\text{red}} = \f{1}{\sqrt 3} \mat{1 & -2 \\ 2 & -1} =: \f{1}{\sqrt 3} M
\end{align*}
and
\begin{align*}
\cR_{\f{\pi}2} f(x) = \sum_m f_m e^{i \pa{R_{\f{\pi}2}^{\text{red}} m} a^* \cdot x} = \sum_m f_{M m} e^{i \f{1}{\sqrt 3} m a^* \cdot x} = \cL f\pa{\f{x}{\sqrt 3}}
\end{align*}
where $\cL$ is the action of $M$ on the Fourier coefficients of $f$.

\subsection{Action of mirror}%
\label{sub:action_of_mirror}

We define $M := \text{diag } \pa{-1,1,-1}$, we have
\begin{align*}
\bbM u(x) := u(M x)
\end{align*}

With the lattice $a$ defined in \eqref{eq:lattice_a}, we obtain

\section{Comparision with existing results}%
\label{sec:comparision_with_existing_results}

From \cite{FanCarZhu19}, we verified that with $T=0$, we have Fig 3(a), with the right energies

\subsection{Reduction of Fourier coefficients in $2d$ to $1d$}%
\label{sub:reduction_of_fourier_coefficients_in_2d_to_1d_}

This is used to compute $V\ind{int}$. We take a function $f$ and define its average
\begin{align*}
g(z) := \f{1}{\ab{\Omega}} \int_\Omega f
\end{align*}
and since
\begin{align*}
\widehat{f}_{0,m_z} = \f{1}{\sqrt{L \ab{\Omega}}} \int_\Omega f(x,z) e^{-i\f{2\pi}{L} m_z z} \d x \d z
\end{align*}
then
\begin{align*}
\widehat{g}_{m_z} = \f{1}{\ab{\Omega} \sqrt{L}} \int_{\Omega \times [0,L]} f(x,z) e^{-i\f{2\pi}{L} m_z z} \d x \d z = \f{\widehat{f}_{0,m_z}}{\sqrt{\ab{\Omega}}}
\end{align*}



\section{Computation of $V\ind{int}$}%
\label{sec:computation_of_vint_}

For $\bs \in \Omega := [0,1] \ba_1 + [0,1] \ba_2$, we denote by $V^{(2)}_{\bs}$ the true Kohn-Sham mean-field potential for the configuration where the two sheets are aligned (no angle), but with the upper one shifted by a vector $\bs$. We set
\begin{align*}
	V_{\rm int, \bs}(z) &:= \f{1}{\ab{\Omega}} \int_{\Omega}  \left( V^{(2)}_{\bs}(\bx, z) - V(\bx, z + \dd) - V(\bx - \bs, z -\dd)   \right) \d \bx \\
	&= \f{1}{\ab{\Omega}} \int_{\Omega}  \left( V^{(2)}_{\bs}(\bx, z) - V(\bx, z + \dd) - V(\bx, z -\dd)   \right) \d \bx \\
    &= \f{1}{\ab{\Omega}^{\f 32}} \sum_{\substack{\bmm \in \Z^2 \\ m_z \in \Z}}  \pa{ \widehat{\pa{V^{(2)}_{\bs}}}_{\bmm,m_z} - \widehat{V}_{\bmm,m_z}e^{i m_z \f{2\pi}{L} \f{d}{2}} -\widehat{V}_{\bmm,m_z} e^{-i m_z \f{2\pi}{L} \f{d}{2}}} \\
    &\qquad \qquad \times \int_{\Omega} e^{i \pa{\bmm \ba^* \cdot \bx + m_z \f{2\pi}{L} z}}\d \bx \\
    &= \f{1}{\sqrt{\ab{\Omega}}}\sum_{\substack{m_z \in \Z}} e^{i  m_z \f{2\pi}{L} z} \pa{ \widehat{\pa{V^{(2)}_{\bs}}}_{0,m_z} - 2 \widehat{V}_{0,m_z} \cos\pa{ m_z \tfrac{\pi d}{L} }}
\end{align*}
and we obtain the Fourier coefficients
\begin{align*}
\pa{\widehat{V_{\rm{int},\bs}}}_{m_z} =\f{1}{\sqrt{\ab{\Omega}}} \pa{\widehat{\pa{V^{(2)}_{\bs}}}_{0,m_z} - 2 \widehat{V}_{0,m_z} \cos\pa{ m_z \tfrac{\pi d}{L} }}
\end{align*}
We then compute
\begin{align*}
V\ind{int}(z) := \f{1}{\ab{\Omega}} \int_\Omega V_{\rm int, \bs}(z) \d \bs = \f{1}{N\ind{int}^2} \sum_{\substack{s_x,s_y \in \intent{1,N\ind{int}}}} V^{\rm{array}}_{\rm int, (s_x,s_y)}(z)
\end{align*}
and finally obtain the Fourier coefficients
\begin{align*}
\boxed{\pa{\widehat{V\ind{int}}}_{m_z} = \f{1}{N\ind{int}^2} \sum_{\substack{s_x,s_y \in \intent{1,N\ind{int}}}} \pa{\widehat{V_{\rm{int},\bs}}}_{m_z}}
\end{align*}
and we expect $V_{\rm{int},\bs}$ not to depend too much on $\bs$, that is we expect that
\begin{align*}
	\delta_{V\ind{int}} &:= \f{\int_{\Omega\times \R} \ab{V_{\rm{int},\bs}(z) - V\ind{int}(z)}^2 \d \bs \d z}{\ab{\Omega} \int_{\R} V\ind{int}(z)^2 \d z} \\
&= \f{\sum_{m_z} \int_{\Omega} \ab{\pa{\widehat{V_{\rm{int},\bs}}}_{m_z} - \pa{\widehat{V\ind{int}}}_{m_z}}^2 \d \bs }{\ab{\Omega} \sum_{m_z}  \pa{\widehat{V\ind{int}}}_{m_z}^2} \\
&= \f{\sum_{s_x,s_y,m_z} \ab{\pa{\widehat{V}_{\rm{int},(s_x,s_y)}}_{m_z} - \pa{\widehat{V\ind{int}}}_{m_z}}^2 }{N\ind{int}^2 \sum_{m_z}  \pa{\widehat{V\ind{int}}}_{m_z}^2}
\end{align*}
is small. We also verify that $V\ind{int}(-z) = V\ind{int}(z)$.



\section{Effective potentials}%
\label{sec:effective_potentials}

We defined
\begin{align*}
\db{ f, g}^{\eta,\eta'}(\bX) :=   \int_{\Omega \times \R} \overline{f}\pa{x -\eta J\bX,z- \eta\dd} g\pa{x - \eta' J \bX, z- \eta'\dd} \d \bx \d z
\end{align*}
and
\begin{align*}
\boxed{\lAngle f, g \rAngle^{\eta,\eta'} := e^{i\pa{\eta'-\eta} \bK \cdot J \bX}\db{ f, g}^{\eta,\eta'}}
\end{align*}
and in particular since $q_1 = -2J K$, then
\begin{align*}
\lAngle f, g \rAngle^{+-} = e^{-iq_1 x}\db{ f, g}^{+-}, \qquad \lAngle f, g \rAngle^{-+} = e^{iq_1 x}\db{ f, g}^{-+}
\end{align*}
 Now we make the approximation
\begin{align*}
\int_{\Omega\times\R} \simeq  \int_{\Omega\times \seg{0,L}}
\end{align*}
The situation is drawn on Figure \ref{fig:z_drawing}. The functions are defined on $[-L/2,L/2]$ but we need to integrate on the common segment, which is $[-\f{L-d}{2},\f{L-d}{2}]$, so on $[-L/2,L/2]$ to recover the initial domain.




Firstly, using the Fourier decomposition \eqref{eq:dec_four},
\begin{align*}
	\db{f,g}^{\eta,\eta'} &=   \sum_{\bmm \in \Z^2} e^{i\pa{\eta-\eta'} ma^* \cdot J \bX} \sum_{m_z \in \Z} e^{i\pa{\eta-\eta'} \f{2\pi}{L} m_z \f{d}{2}} \overline{\widehat{f}_{m,m_z}} \widehat{g}_{m,m_z}\\
& = \sum_{\bmm \in \Z^2} \f{e^{i\pa{\eta-\eta'} ma^* \cdot J \bX}}{\sqom} C_{-\bmm}
\end{align*}
where
\begin{align*}
\boxed{C_\bmm := \sqom \sum_{m_z \in \Z} e^{i\pa{\eta-\eta'} \f{d\pi}{L} m_z} \overline{\widehat{f}_{-m,m_z}} \widehat{g}_{-m,m_z}}.
\end{align*}

We have $\db{f,g}^{++} = \db{f,g}^{--} = \ps{f,g} = \sum_{m,m_z} \overline{\widehat{f}_{m,m_z}} \widehat{g}_{m,m_z}$.

 We also define, for $\eta \in \{-1,1\}$,
\begin{align*}
C^\eta_\bmm := \sqom\sum_{m_z \in \Z} e^{\eta i2 \f{d\pi}{L} m_z} \overline{\widehat{f}_{-\eta m,m_z}} \widehat{g}_{-\eta m,m_z}
\end{align*}


We have $a^*\ind{M} = 2 J a^*$ hence $2 ma^* \cdot JX = -m a\ind{M}^* \cdot X$ and
\begin{align*}
\db{f,g}^{+-} = \sum_{\bmm \in \Z^2} \f{e^{ima^*\ind{M} \cdot \bX}}{\sqom} C_{\bmm}^+, \qquad \db{f,g}^{-+} = \sum_{\bmm \in \Z^2} \f{e^{ima^*\ind{M} \cdot \bX}}{\sqom} C_{\bmm}^-
\end{align*}
Now, since $q_{2,3} - q_1 = a^*_{\text{M},j}$, we know that
\begin{align*}
G(x) &= e^{-iq_1 x} \pa{1 + e^{-i a_{\text{M} ,1}^* x} +e^{-i a_{\text{M},2}^* x}} \\
F(x) &= e^{-iq_1 x} \pa{1 + \omega^2 e^{-i a_{\text{M} ,1}^* x} + \omega e^{-i a_{\text{M},2}^* x}}
\end{align*}

We have $\bbV^{1,1} \simeq w\ind{AA} G$ so $\ps{G,\bbV} \simeq w\ind{AA} \int_{\Omega\ind{M}} \ab{G}^2 = 3 \ab{\Omega\ind{M}} w\ind{AA}$ and hence 
\begin{align*}
	w\ind{AA} &\simeq \f{\ps{G,\bbV^{1,1}}}{3 \ab{\Omega\ind{M}}} = \f{1}{3 \sqom}\pa{\widehat{\bbV}^{1,1}_{0,0} +\widehat{\bbV}^{1,1}_{-1,0} + \widehat{\bbV}^{1,1}_{0,-1}} \\
	w\ind{AB} &\simeq \f{\ps{F,\bbV^{1,2}}}{3 \ab{\Omega\ind{M}}} = \f{1}{3 \sqom}\pa{\widehat{\bbV}^{1,2}_{0,0} + \omega\widehat{\bbV}^{1,2}_{-1,0} + \omega^2 \widehat{\bbV}^{1,2}_{0,-1}}
\end{align*}

Then,
\begin{align*}
\lAngle f,g \rAngle ^{\eta,\eta'} = e^{i\pa{\eta-\eta'} \bK \cdot J \bX}\db{f,g}^{\eta,\eta'} =  \sum_{\bmm \in \Z^2} e^{i\pa{\eta-\eta'} \pa{m+m_K}a^* \cdot J \bX} C_\bmm
\end{align*}

Moreover, 
\begin{align*}
	\cA &= \f{1}{2} J \pa{-i\na} \Sigma= \f{1}{2} J \pa{-i\na} \pa{e^{-iq_1 x} \db{u_j,u_{j'}}^{+-}} \\
	    &= e^{-iq_1 x}\f 12 \sum_{m \in \Z^2} J \pa{ma_{\text{M}}^*} \f{e^{ima^*\ind{M} \cdot \bX}}{\sqom} C_{\bmm}^{+,\Sigma} - \f 12 \Sigma J q_1 ,\\
	    & \underset{\substack{q_1 = -2 JK}}{=} \; e^{-iq_1 x} \sum_{m \in \Z^2} \f{e^{ima^*\ind{M} \cdot \bX}}{\sqom} C_{\bmm}^{+,\cA} - \Sigma K,\\
C_{m}^{+,\cA} &:= \f 12 J \pa{ma_{\text{M}}^*}C_{\bmm}^{+,\Sigma}
\end{align*}

To plot, we will need to shift the exponent $e^{-iq_1 x} = e^{\f{i}{3}\pa{a_{\text{M},1}^*+a_{\text{M},2}^*}x}$

{\color{red}{TO RECTIFY}}

Hence
\begin{align*}
\boxed{\db{f,g}^{+-}\pa{-\tfrac{3}{2} J \bX} =  \sum_{\bmm \in \Z^2} e^{i 3 ma^* \cdot \bX} C^+_\bmm=  \sum_{\bmm \in \Z^2} e^{i m a^* \cdot \bX} C^+_{\f{\bmm}{3}},}
\end{align*}
and
\begin{align*}
\boxed{\lAngle f,g \rAngle^{+-}\pa{-\tfrac{3}{2} J\bX} =  \sum_{\bmm \in \Z^2} e^{i 3 \pa{m+m_k}a^* \cdot \bX} C^+_\bmm =  \sum_{\bmm \in \Z^2} e^{i m a^* \cdot \bX} C^+_{\f{\bmm-3\bmm_K}{3}}}
\end{align*}
where $C_{\f{\bmm}{n}} := 0$ if $n$ does not divide $m_1$ and $m_2$. Numerically, there is no loss of information since all $C_m$'s are taken into account if the ``ecut'' is large enough.

Similarly
\begin{align*}
\db{f,g}^{-+}\pa{-\tfrac{3}{2} J \bX} =  \sum_{\bmm \in \Z^2} e^{-i 3 ma^* \cdot \bX} C^-_\bmm=  \sum_{\bmm \in \Z^2} e^{i m a^* \cdot \bX} C^-_{-\f{\bmm}{3}},
\end{align*}

For the potentials, we finally need to implement
\begin{multline*}
\bbW_{j,j'}^+ = \db{\overline{u}_j u_{j'}, V}^{+-}, \qquad \bbW_{j,j'}^- = \db{\overline{u}_j u_{j'}, V}^{-+},\\
\bbV_{j,j'} = \lAngle \pa{V+V\ind{int}} u_j , u_{j'} \rAngle^{+-}
\end{multline*}


If $f(z) = \ep f(-z)$, then $\widehat{f}_{-m_z} = \ep \widehat{f}_{m_z}$, from this we see that $\overline{C_\bmm^{u_{j'},u_{j}}} = C_\bmm^{u_{j},u_{j'}}$ and hence $\bbV(-X)^* = \bbV(X)$


\subsection{Magnetic term}%


As for the magnetic term, we have
\begin{align*}
\pa{-i\na_\bx + \bK} g = \sum_{\bmm,m_z} \pa{\bmm + \bmm_K} \ba^* e^{i\pa{\bmm \ba^* \cdot \bx + m_z \f{2\pi}L z}} \widehat{f}_{\bmm,m_z}
\end{align*}
so
\begin{align*}
	\lAngle &f, (-i\nabla_\bx+\bK) g \rAngle^{+-}(\bX) =  \sum_{\bmm \in \Z^2} \pa{\bmm + \bmm_K}\ba^* \; C_\bmm e^{2i \pa{\bmm + \bmm_K} \ba^* \cdot J\bX} 
\end{align*}
and
\begin{align*}
\lAngle f, (-i\nabla_\bx+\bK) g \rAngle^{+-}\pa{-\tfrac{3}{2}J\bX} =  \sum_{\bmm \in \Z^2} \pa{\bmm + \bmm_K}\ba^* \; C_\bmm e^{i 3 \pa{\bmm + \bmm_K} \ba^* \cdot \bX}
\end{align*}
so
\begin{align*}
\boxed{\lAngle f, (-i\nabla_\bx+\bK) g \rAngle^{+-}\pa{-\tfrac{3}{2}J\bX} = \f {1}3 \sum_{\bmm \in \Z^2} \bmm \ba^* \; C_{\f{\bmm - 3\bmm_K}{3}} e^{i \bmm \ba^* \cdot \bX} }
\end{align*}
so we can implement
\begin{align*}
\bm{\cA}_{j,j'}\pa{-\tfrac{3}{2}J\bX} = \lAngle u_j, (-i\nabla_\bx+\bK) u_{j'} \rAngle^{+-}\pa{-\tfrac{3}{2}J\bX}
\end{align*}

\subsection{$\bbW$'s $V\ind{int}$ term}%
\label{sub:_bbw_s_vint_term}
We write $V\ind{int}(z) = \f{1}{\sqrt{L}} \sum_{m_z \in \Z} \widehat{V}\ind{int}^{m_z} e^{i \f{2\pi}L m_z z}$ hence
\begin{align*}
	\ps{u_j, V\ind{int} u_{j'}} &= \f{1}{L^{\f 32}} \sum_{\substack{\bmm \in \Z^2 \\ m_z,m_z',M_z \in \Z}} \pa{\overline{\widehat{u}}_j}_{\bmm,m_z} \pa{\widehat{u}_{j'}}_{\bmm,m_z'}\pa{\widehat{V\ind{int}}}_{M_z} \int_z e^{iz \f{2\pi}{L} \pa{M_z + m_z'-m_z}} \\
& = \f{1}{\sqrt{L}} \sum_{\substack{\bmm \in \Z^2 \\ m_z,m_z' \in \Z}} \pa{\overline{\widehat{u}}_j}_{\bmm,m_z} \pa{\widehat{u}_{j'}}_{\bmm,m_z'}\pa{\widehat{V\ind{int}}}_{m_z-m_z'} 
\end{align*}
and the matrix $M_{j,j'} := \ps{u_j, V\ind{int} u_{j'}}$ is such that $M^* = M$ and $M_{11} = M_{22}$.

In the function $\bbV(X) = \ps{u_j,V u_i}(X)$, when $V \rightarrow V+ V\ind{int}$, we have 
\begin{align*}
\widetilde{\bbV}(X) = \ps{u_j,(V+V\ind{int}) u_i}(X) = \bbV(X) + \ps{u_j,V\ind{int} u_i}
\end{align*}
but at the level of Fourier coefficients,
\begin{align*}
\widehat{\widetilde{\bbV}}_0 = \widehat{\bbV}_0 + \f{\ps{u_j,V\ind{int} u_i}}{\sqrt{\ab{\Omega}}}
\end{align*}
so when we add it to the Fourier Hamiltonian, we should not forget to divide by $\sqrt{\ab{\Omega}}$

\subsection{Adding a constant}%
\label{sub:adding_a_constant}
We have
\begin{align*}
g(x) := f(x) + c \qquad \implies \qquad \widehat{g}_0 = \widehat{f}_0 + \sqom c
\end{align*}

\subsection{Substracting the mean of $\bbW^+$}%
\label{sub:substracting_the_mean_of_bbw_}

To do this, we do it for a function $f$, 

\begin{align*}
\frac{1}{\ab{\Omega}} \int f = \f{\widehat{f}_{0}}{\sqrt{\ab{\Omega}}}
\end{align*}
hence
\begin{align*}
g(x) := f(x) -  \frac{1}{\ab{\Omega}} \int f \qquad \implies \qquad \widehat{g}_0 = 0
\end{align*}


\subsection{$V\ind{int}^{3d}$}%
\label{sub:_v__int}

We have $V\ind{int}^{3d}(x,z) := V\ind{int}(z)$ hence $\pa{\widehat{V}\ind{int}^{3d}}_{m,m_z} = \sqom \pa{\widehat{V}\ind{int}}_{m_z}$


\section{Form}%
\label{sec:form}

If we have
\begin{align*}
	T(x) = \mat{e^{-iq_1 x} g(x) & e^{-iq_1 x} f(x) \\ e^{-iq_1 x}\overline{f}(-x) & e^{-iq_1 x}g(x)} = e^{-iq_1 x}\mat{ g(x) &  f(x) \\ \overline{f}(-x) & g(x)},
\end{align*}
then
\begin{align*}
T^*(x) = e^{iq_1 x}\mat{ \overline{g}(x) &  f(-x) \\ \overline{f}(x) & \overline{g}(x)}
\end{align*}
and $\bbW$ does not have an exponent $e^{-iq_1 x}$.

\subsection{From $T(x)$ to $T(3x)$}%
\label{sub:from_t_x_to_t_3x_}

With $T(x) = e^{-iq_1 x} \sum T_m e^{i m a_M^* x}$, we have
\begin{align*}
\mathcal{T}(x) := T(3x) = e^{i\pa{a_{1,M}^* + a_{2,M}^*}x}  \sum T_m e^{i m a_M^* 3x} = \sum_m T_{\f{m-\mat{1\\1}}{3}} e^{ima_M^* x}
\end{align*}


\section{BM configuration}%
\label{sec:bm_configuration}


From \cite{BecEmbWitZwo21}, the BM Hamiltonian is
\begin{align*}
	H = \mat{-i\sigma \na & T^c(x) \\ T^c(x)^* & -i \sigma \na},
\end{align*}
where
\begin{align*}
\boxed{T_1 = \mat{w_0 & w_1 \\ w_1 & w_0}, \quad  T_2 = \mat{w_0 &  w_1e^{-i\phi} \\  w_1e^{i\phi} & w_0}, \quad T_3 = \mat{w_0 &  w_1e^{i\phi} \\  w_1e^{-i\phi} & w_0}}
\end{align*}
and where, for $x \in \R^2$,
\begin{align*}
T^c(x) := \sum_{j=1}^3 T_j e^{-iq^c_j \cdot x} = \sum_{j=1}^3 T_j e^{iq_j a^*\cdot x}, \qquad \widehat{T}_p = \sum_{j=1}^{3} T_j \delta_{p,q_j^c}
\end{align*}
and
\begin{multline*}
q^c_1 = \f{4\pi}{a\sqrt{3}} \mat{1 \\ 0} = a_1^* + a_2^*, \\
q^c_2 = \f{4\pi}{a\sqrt{3}} \mat{-\f 12 \\ \f{\sqrt{3}}{2}} = -a_2^*, \qquad q^c_3 = \f{4\pi}{a\sqrt{3}} \mat{-\f 12 \\ -\f{\sqrt{3}}{2}} = -a_1^*,
\end{multline*}
where we took rotated $q_j^c$'s by $J$ with respect to \cite{BecEmbWitZwo21}, and with a rescaling of $\f{4\pi}{a\sqrt{3}}$.

We define the reduced dual vectors $q_j := -\cM^* q^c_j /2\pi$ so
\begin{align*}
T(x) = T^c(\cM x) = \sum_{j=1}^3 T_j e^{-ix\cdot \cM^* q_j^c}= \sum_{j=1}^3 T_j e^{i2\pi x\cdot q_j}
\end{align*}
and we compute
\begin{align*}
\boxed{q_1 = \mat{1 \\ 1}, \qquad q_{2} = \mat{0 \\ -1}, \qquad q_{3} = \mat{-1 \\ 0}}
\end{align*}
Or
\begin{align*}
\boxed{T(x) = \sum_{j=1}^3 T_j e^{i q_j a^* \cdot x}}
\end{align*}

Since $T_j^* = T_j$, then $T(-x)^* = T(x)$




\section{Operators in basis}%
\label{sec:operators_in_basis}


\subsection{Goal}%
\label{sub:goal}
Our goal is to study the eigenvalue equation
\begin{align*}
\boxed{\cH \p = \ep_\theta \cS E \p}
\end{align*}
remark that energies have to be rescaled by $\ep_\theta$ ! The operator $\cS$ is Hermitian and positive and
\begin{align*}
\boxed{\cH := \f{1}{\ep_\theta} \cV + c_\theta T + \ep_\theta T^{(1)}}
\end{align*}
where
\begin{align*}
T &:=  v_{\rm F} \left( \begin{array}{cc} \bm\sigma \cdot (-i \nabla)  &  \bm{\mathcal A} \cdot (-i\nabla)   \\  \bm{\mathcal A}^* \cdot (-i\nabla) &  \bm\sigma \cdot (-i \nabla)  \end{array} \right),  \\
T^{(1)} &:= v_{\rm F} \left( \begin{array}{cc}  -  \bm\sigma \cdot J (-i \nabla) - \frac 12  \Delta &  \bm{\mathcal A} \cdot J (-i\nabla) - \frac 12  \Sigma \Delta  \\  \bm{\mathcal A}^*  \cdot J (-i\nabla) - \frac 12 \Sigma^* \Delta &  \bm\sigma \cdot J (-i \nabla) - \frac 12   \Delta \end{array} \right), \\
\cV &:=  \left( \begin{array}{cc}  \bbW &   \bbV \\   \bbV^* &   \bbW \end{array} \right),
\end{align*}
and their Bloch transform becomes
\begin{align*}
T_k &:=  v_{\rm F} \left( \begin{array}{cc} \bm\sigma \cdot (-i\na +k)  &  \bm{\mathcal A} \cdot (-i\na +k)   \\  \bm{\mathcal A}^* \cdot (-i\na +k) &  \bm\sigma \cdot (-i\na +k)  \end{array} \right),  \\
\hspace{-5cm} T_k^{(1)} &:= v_{\rm F} \left( \begin{array}{cc}  -  \bm\sigma \cdot J (-i\na +k) +\f 12  (-i\na +k)^2 &  \bm{\mathcal A} \cdot J (-i\na +k) +\f 12  \Sigma (-i\na +k)^2  \\  \bm{\mathcal A}^*  \cdot J (-i\na +k) +\f 12 \Sigma^* (-i\na +k)^2 &  \bm\sigma \cdot J (-i\na +k) +\f 12   (-i\na +k)^2 \end{array} \right)
\end{align*}
and we want the middle of the spectrum of
\begin{align*}
\cH_k := \cS^{-\f 12} \pa{\f{1}{\ep_\theta} \cV + c_\theta T_k + \ep_\theta T^{(1)}_k}\cS^{-\f 12}
\end{align*}


\subsection{Basis}%
\label{sub:basis}

We define $e_m := \f{1}{\sqrt{\ab{\Omega}}} e^{i m a^* \cdot x}$, and
\begin{align*}
e_{\alpha,m} := e_\alpha \otimes e_m = e_\alpha \f{e^{ima^*\cdot x}}{\sqrt{\ab{\Omega}}}, \qquad \text{where } e_1 := \mat{1 \\ 0 \\ 0 \\ 0},\dots
\end{align*}

\subsection{Multiplication-derivation operators}%

For $A = (A_1,A_2)$ and $A_j = \sum_\ell \pa{\widehat{A_j}}_\ell e^{i \ell a^*\cdot x}$, we have
\begin{multline*}
\ps{e_n, A \cdot (-i\na +k) e_m} = \sum_{\ell} \pa{\widehat{A_1}}_\ell \pa{ma^* + k}_1\ps{e_n, e^{i\ell a^*\cdot x} e_m} \\
+ \pa{\widehat{A_2}}_\ell \pa{ma^* + k}_2\ps{e_n, e^{i\ell a^*\cdot x} e_m} \\
= \pa{\widehat{A_1}}_{n-m} \pa{ma^* + k}_1 + \pa{\widehat{A_2}}_{n-m} \pa{ma^* + k}_2 = \widehat{A}_{n-m} \cdot \pa{ma^*+k}
\end{multline*}

For $V = \sum_\ell \widehat{V}_\ell e^{i\ell a^*x}$, we have $\ps{e_n,V e_m} = \widehat{V}_{n-m}$ and
\begin{align*}
\ps{e_n,V (-i\na + k)^2 e_m} =  \pa{ma^*+k}^2\widehat{V}_{n-m}
\end{align*}
\subsection{On-diagonal potential}%
\label{sub:on_diagonal_potential}



For a general $W^\pm = \sum_m W^\pm_m e^{im a^* \cdot x}$, we have
\begin{align*}
	\ps{e_{\alpha,n},\mat{W^+ & 0 \\ 0 & W^-} e_{\beta,m}} = \delta_{\alpha \in \acs{1,2}}^{\beta \in \acs{1,2}}\pa{W^+_{n-m}}_{\alpha_1 \beta_1} + \delta_{\alpha \in \acs{3,4}}^{\beta \in \acs{3,4}}\pa{W^-_{n-m}}_{\alpha_2 \beta_2}
\end{align*}

\subsection{Off-diagonal potential}%
\label{sub:off_diagonal_potential}



For a general $V = \sum_m V_m e^{im a^* \cdot x}$, we have $V^* = \sum_m V_m^* e^{-im a^* \cdot x}$ and
\begin{align*}
	M_{IJ} & := \ps{e_{\alpha,n}, \mat{0 & V \\ V^* & 0} e_{\beta,m}} \\
	       &= \sum_{k} \pa{\delta_{\alpha \in \acs{1,2}}^{\beta \in \acs{3,4}} \delta_{m+k-n} \ps{e_{\alpha_1},V_k e_{\beta_2}} + \delta_{\alpha \in \acs{3,4}}^{\beta \in \acs{1,2}}\delta_{m-k-n} \ps{e_{\alpha_2},V_k^* e_{\beta_1}}} \\
	       &=   \delta_{\alpha \in \acs{1,2}}^{\beta \in \acs{3,4}}\ps{e_{\alpha_1},V_{n-m} e_{\beta_2}} + \delta_{\alpha \in \acs{3,4}}^{\beta \in \acs{1,2}}\ps{e_{\alpha_2},V_{m-n}^* e_{\beta_1}} \\
	       &=     \delta_{\alpha \in \acs{1,2}}^{\beta \in \acs{3,4}}\pa{V_{n-m}}_{\alpha_1 \beta_2} + \delta_{\alpha \in \acs{3,4}}^{\beta \in \acs{1,2}} \overline{\pa{V_{m-n}}_{\beta_1\alpha_2} }
\end{align*}
and $M$ is also Hermitian.

\subsection{Off-diagonal magnetic term}%
\label{sub:off_diagonal_magnetic_term}



For a general $A = \mat{A_1 \\ A_2}$, $A_j = \sum_\ell  \pa{A_j}_\ell  e^{i\ell a^*\cdot x}$, we have $A_j^* = \sum_\ell  \pa{A_j}^*_\ell  e^{-i\ell a^*\cdot x}$ and we compute
\begin{multline*}
\ps{e_{\alpha,n}, \mat{0 & A \cdot \pa{-i\na +k} \\A^* \cdot \pa{-i\na +k}  & 0} e_{\beta,m}} \\
= \delta_{\alpha \in \acs{1,2}}^{\beta \in \acs{3,4}} \pa{ \pa{ma^* +k}_1 \pa{\pa{A_1}_{n-m}}_{\alpha_1 \beta_2} + \pa{ma^* +k}_2 \pa{\pa{A_2}_{n-m}}_{\alpha_1 \beta_2}}\\
+ \delta_{\alpha \in \acs{3,4}}^{\beta \in \acs{1,2}} \pa{ \pa{ma^* +k}_1 \overline{\pa{\pa{A_1}_{m-n}}_{\beta_1 \alpha_2}} + \pa{ma^* +k}_2 \overline{\pa{\pa{A_2}_{m-n}}_{\beta_1 \alpha_2}}}
\end{multline*}


\subsection{Dirac operator}%
\label{sub:dirac_operator}

We have 
\begin{align*}
	\sigma \cdot \pa{-i\na + k} &= \sigma_1 \pa{-i\partial_1 + k_1} + \sigma_2 \pa{-i\partial_2 + k_2} \\
				    & = \mat{0 & -i\pa{\partial_1 -i\partial_2} + \overline{k_\C} \\ -i\pa{\partial_1 +i\partial_2} + k_\C & 0}
\end{align*}
 where
\begin{align*}
	\sigma_1 = \mat{0 & 1 \\ 1 & 0} \qquad \sigma_2 = \mat{0 & -i \\ i & 0}
\end{align*}
so, with $k_\C := k_1 +ik_2$,
\begin{align*}
\sigma \cdot \pa{-i\na + k} \mat{1 \\ 0} e_m = \pa{ma^* + k}_\C \mat{0 \\ 1} e_m \\
\sigma \cdot \pa{-i\na + k} \mat{0 \\ 1} e_m = \overline{\pa{ma^* + k}_\C} \mat{1 \\ 0} e_m
\end{align*}


Then
\begin{align*}
	\mat{\sigma \cdot \pa{-i\na + k} & 0 \\ 0 & \sigma \cdot \pa{-i\na + k}} e_{1,m} &= \pa{ma^* + k}_\C  e_{2,m} \\
	\mat{\sigma \cdot \pa{-i\na + k} & 0 \\ 0 & \sigma \cdot \pa{-i\na + k}} e_{2,m} &= \overline{\pa{ma^* + k}_\C}  e_{1,m} \\
	\mat{\sigma \cdot \pa{-i\na + k} & 0 \\ 0 & \sigma \cdot \pa{-i\na + k}} e_{3,m} &= \pa{ma^* + k}_\C  e_{4,m} \\
	\mat{\sigma \cdot \pa{-i\na + k} & 0 \\ 0 & \sigma \cdot \pa{-i\na + k}} e_{4,m} &= \overline{\pa{ma^* + k}_\C}  e_{3,m}
\end{align*}





We know that $e^{-ikx} \pa{-i\na} e^{ikx} \cdot = -i\na + k$ hence
\begin{align*}
e^{-ikx} \pa{-\f 12 \Delta} e^{ikx} \cdot = \f 12 \pa{-i\na + k}^2
\end{align*}
and with $f(x) = \sum_m \widehat{f}_m e^{ima^*x}$
\begin{align*}
\pa{-i\na +k} f = \sum_m  \pa{ma^* +k} \widehat{f}_m e^{ima^*x},
\end{align*}
so
\begin{align*}
\f 12 \pa{-i\na +k}^2 f = \sum_m  \f 12 \pa{ma^* +k}^2 \widehat{f}_m e^{ima^*x}
\end{align*}

We have
\begin{align*}
\ps{e_{\alpha,n}, \f 12 \pa{-i\na +k}^2 e_{\beta,m}} = \f 12 \pa{ma^* + k}^2 \delta_{\alpha,\beta} \delta_{m-n}
\end{align*}

We have
\begin{align*}
	\sigma \cdot k = \mat{0 & \overline{k_\C} \\ k_\C & 0}, \qquad \pa{J k}_\C = i k_\C, \qquad \sigma \cdot J k = \mat{0 & -i \overline{k_\C} \\ i k_\C & 0}
\end{align*}
so
\begin{align*}
	\mat{-\sigma \cdot J \pa{-i\na + k} & 0 \\ 0 & \sigma \cdot J \pa{-i\na + k}} e_{1,m} &= -i\pa{ma^* + k}_\C  e_{2,m} \\
	\mat{-\sigma \cdot J \pa{-i\na + k} & 0 \\ 0 & \sigma \cdot J \pa{-i\na + k}} e_{2,m} &= i \; \overline{\pa{ma^* + k}_\C} \; e_{1,m} \\
	\mat{-\sigma \cdot J \pa{-i\na + k} & 0 \\ 0 & \sigma \cdot J \pa{-i\na + k}} e_{3,m} &= i\pa{ma^* + k}_\C  e_{4,m} \\
	\mat{-\sigma \cdot J \pa{-i\na + k} & 0 \\ 0 & \sigma \cdot J \pa{-i\na + k}} e_{4,m} &= -i \; \overline{\pa{ma^* + k}_\C} \; e_{3,m}
\end{align*}

For a general $V = \sum_m \widehat{V}_m e^{ima^*\cdot x}$, we have $V^* = \sum_m \widehat{V}^*_m e^{-ima^*\cdot x}$ and we compute
\begin{multline*}
\ps{e_{\alpha,n}, \mat{0 & V  \pa{-i\na +k}^2 \\V^*  \pa{-i\na +k}^2  & 0} e_{\beta,m}} \\
=  \pa{ma^* +k}^2 \pa{\delta_{\alpha \in \acs{1,2}}^{\beta \in \acs{3,4}} \pa{\widehat{V}_{n-m}}_{\alpha_1 \beta_2} + \delta_{\alpha \in \acs{3,4}}^{\beta \in \acs{1,2}}   \overline{\pa{\widehat{V}_{m-n}}_{\beta_1 \alpha_2}}}
\end{multline*}

\section{Renormalization of the equation}%
\label{sec:renormalization_of_the_equation}

We know that
\begin{align*}
\pa{-i\na +k + A(x) \pa{-i\na} +v(x)} \p = E \p
\end{align*}
with $x=\lambda y$, we define $\phi(y) := \p(\lambda y)$ and
\begin{align*}
\pa{\pa{\pa{-i\na +k} + A(\lambda y) \pa{-i\na} +v(\lambda y)} \p}(\lambda y) = E \p(\lambda y)
\end{align*}
but $\pa{\na \p}(\lambda y) = \f{1}{\lambda} \na \phi(y)$, so
\begin{align*}
\pa{\f{-i\na}{\lambda}+k+ \f{A(\lambda y)}{\lambda} \pa{-i\na} + v(\lambda y)} \phi =   E \phi
\end{align*}
We enter $V\pa{\f{3}{2}J X}$ for each potential $V$, hence we need to apply a coefficient $\f{2}{3}$ to each derivation operator.



\section{Symmetries}%
\label{sec:symmetries}


\subsection{Particle-hole}%
\label{sub:particle_hole}

We define
\begin{align*}
\cS u(x) := i \mat{0 & -\1_{2\times 2} \\ \1_{2\times 2} & 0} u(-x)
\end{align*}
We have
\begin{align*}
\cS \mat{0 & B \\ B^* & 0} \cS = -\mat{0 & B^*(-x) \\ B(-x) & 0}
\end{align*}
We have $T(-x)^* = T(x)$ hence we should have that
\begin{align*}
\cS H \cS = - H
\end{align*}

We compute
\begin{align*}
	\cS_{IJ} &= \ps{e_{\alpha,n},\cS e_{\beta,m}} = i \ps{e_{\alpha,n}, \mat{-e_{\beta_2,-m} \\ e_{\beta_1,-m}}} \\
&= i \delta_{m+n} \pa{\delta_{\alpha \in \acs{3,4}}^{\beta \in \acs{1,2}} \delta_{\beta_1-\alpha_2} -\delta_{\alpha \in \acs{1,2}}^{\beta \in \acs{3,4}}\delta_{\beta_2-\alpha_1}}
\end{align*}



For any function $B$ and any vector function $\bm{A}$, we have
\begin{align*}
	\cS \mat{0 & B(\bX) \\ B^*(\bX) & 0} \cS &= -\mat{0 & B^*(-\bX) \\ B(-\bX) & 0} \\
	\cS \mat{0 & B(\bX) \Delta \\ B^*(\bX)\Delta & 0} \cS &= -\mat{0 & B^*(-\bX)\Delta \\ B(-\bX)\Delta & 0} \\
	\cS\mat{0 & i\bm{A}(\bX) \cdot \na \\ i\bm{A}(\bX)^* \cdot \na & 0} \cS &= \mat{0 & i \bm{A}(-\bX)^* \cdot \na \\ i\bm{A}(-\bX) \cdot \na & 0},
\end{align*}
we also compute that
\begin{align*}
	\cS \mat{\sigma \cdot \na & 0 \\ 0 & \sigma \cdot \na} \cS &= - \mat{\sigma \cdot \na & 0 \\ 0 & \sigma \cdot \na},
\end{align*}
hence if the operator $\Gamma$ is a linear combination of the terms
\begin{multline*}
\mat{\sigma \cdot \pa{-i\na} & 0 \\ 0 & \sigma \cdot \pa{-i\na}}, \mat{\sigma \cdot J\pa{-i\na} & 0 \\ 0 & \sigma \cdot J\pa{-i\na}}, \\
\mat{0 & \bbV \\ \bbV^* & 0}, \mat{0 & \Sigma \\ \Sigma^* & 0}, \mat{0 & \Sigma \Delta	 \\ \Sigma^* \Delta & 0}
\end{multline*}
it satisfies the symmetry $\cS \Gamma \cS = - \Gamma$, and those are the particle-hole symmetric terms of our effective Hamiltonian. However, if $\Gamma$ is a linear combination of the operators
\begin{align*}
	\mat{0 & \bm{\cA}\cdot\pa{-i\na} \\ \bm{\cA}^* \cdot \pa{-i\na} & 0}, \mat{0 & \bm{\cA}\cdot J\pa{-i\na} \\ \bm{\cA}^* \cdot J\pa{-i\na} & 0},  \\
	\mat{ -\f 12 \Delta & 0 \\ 0 & -\f 12 \Delta}, \mat{\bbW & 0 \\ 0 & \bbW^*}, \mat{\1_{2\times 2} & 0 \\ 0 & \1_{2\times 2}}
\end{align*}
of the effective Hamiltonian $\cH_{d,\theta}$, it satisfies $\cS \Gamma \cS = \Gamma$ and hence break the particle-hole symmetry.

But now we also compute that
\begin{multline*}
	\cS \mat{k & 0 \\ 0 & k} \cS = k, \\
	\cS \mat{\sigma \pa{-i\na +k} & 0 \\ 0 & \sigma\pa{-i\na +k}} \cS = -\mat{\sigma\pa{-i\na -k} & 0 \\ 0 & \sigma\pa{-i\na -k}}
\end{multline*}

\subsection{Mirror}%
\label{sub:mirror}

First, for any function $B$, we have $\sigma_1 B^* \sigma_1 = \mat{\overline{B_{22}} & \overline{B_{12}} \\ \overline{B_{21}} & \overline{B_{11}}}$.


The mirror operator for the BM Hamiltonian is
\begin{align*}
	\cM u(\bX) := \mat{0 & \sigma_1 \\ \sigma_1 & 0} u(\overline{\bX})
\end{align*}
where $\overline{\bX} := (X_1,-X_2) =: M \bX$, it satisfies $\cM = \cM^{-1} = \cM^*$.


Next,
\begin{align*}
	\cM \mat{0 & B(\bX) \\ B(\bX)^* & 0} \cM = \mat{0 & \sigma_1 B^*(\overline{\bX}) \sigma_1 \\ \sigma_1 B(\overline{\bX}) \sigma_1 & 0}%= \mat{0 & \sigma_1 B(\bX)^* \sigma_1 \\ \sigma_1 B(\bX) \sigma_1 & 0}
\end{align*}

In cartesian coordinates, we have 
\begin{align*}
T(M \bX) = \sum_{j=1}^3 T_j e^{i  x \cdot M^*q^c_j} = \sum_{j=1}^3 T_j e^{i x \cdot Mq^c_j}
\end{align*}
because $M^* = M$. But
\begin{multline*}
	\sigma_1 T^*(M \bX) \sigma_1 = \mat{ w_0 \pa{\sum_{j=1}^3 e^{i x \cdot Mq_j}} & w_1\pa{e^{ix \cdot M q_1} + e^{i\phi} e^{ix Mq_2} + e^{i2\phi} e^{ix \cdot Mq_3}} \\ \cdot & \cdot} \\
	= \mat{ w_0 \pa{\sum_{j=1}^3 e^{i x \cdot q_j}} & w_1\pa{e^{ix \cdot q_1} + e^{-i\phi} e^{ix q_2} + e^{-i2\phi} e^{ix \cdot q_3}} \\ \cdot & \cdot} = T(\bX)
\end{multline*}
where we used that $M q_1^c = q_1^c$, $M q_2^c = q_3^c$ and $M q_3^c = q_2^c$.

We search the action on reduced Fourier coefficients. We have
\begin{align*}
f(M x) = \sum_m e^{i x \cdot M\pa{m a^*}} = \sum_m e^{i x \cdot \pa{M^r m}a^*}
\end{align*}
where $M = \mat{1 & 0 \\ 0 & -1}$,
\begin{align*}
M^r = S^{-1} M S = \cM^* M \pa{\cM^*}^{-1} = \sigma_1
\end{align*}


\section{Non Local term}%
\label{sec:non_local_term}

From the theoretical investigations, we have
\begin{align*}
F^{\eta,j,s}(\bX) := \int_{\R^3} \overline{\vp_{\text{Bl},s}(\by,z)} \Phi_j\pa{\by +\ba_s -2\eta J \bX,z-\eta d} \d \by \d z
\end{align*}
and
\begin{align*}
\bbW_{\text{nl},-1}^\eta\pa{\bX}_{jj'} := \f{v_0}{\ab{\Omega}} \sum_{s \in \{1,2\}} \overline{F^{\eta,j,s}(\bX)} F^{\eta,j',s}(\bX).
\end{align*}

Since $\vp_{\text{Bl},s}$ is localized, we periodize it and we make the approximation
\begin{align*}
	F^{\eta,j,s}(\bX) &\simeq \int_{\Omega \times [0,L]} \overline{\vp_{\text{Bl},s}(\by,z)} \Phi_j\pa{\by +\ba_s -2\eta J \bX,z-\eta d} \d \by \d z \\
&=\int_{\Omega \times [0,L]} \overline{\vp_{s}(\by,z)} u_j\pa{\by +\ba_s -2\eta J \bX,z-\eta d} \d \by \d z
\end{align*}
and we define $\vp$ such that $\vp_{\text{Bl},s} = e^{i \bK \by} \vp_s$, because it is $\widehat{\vp_s}$ which is stored by DFTK, so
\begin{align*}
\vp_{s}(\by,z) = \sum_{m, m_z} \f{e^{i\pa{m a^* \by + m_z \f{2\pi}{L}  z}}}{\Gamma} \widehat{\vp}_{s,\bmm,m_z}, \qquad u_j(\by,z) = \sum_{\bmm, m_z} \f{e^{i\pa{\bmm \by + \f{2\pi}{L} m_z z}}}{\Gamma} \widehat{\pa{u_j}}_{\bmm,m_z}
\end{align*}
where $\bK$ is the Dirac point, thus
\begin{align*}
F^{\eta,j,s}(\bX) &=  \sum_{\bmm, m_z} e^{i\pa{\bmm \ba^* \pa{\ba_s - 2\eta J \bX} - \eta \f{2\pi}{L} m_z d}} \overline{\widehat{\vp}}_{s,\bmm,m_z} \widehat{\pa{u_j}}_{\bmm,m_z} \\
		  &=  \sum_{\bmm, m_z} e^{i\pa{\bmm \ba\ind{M}^* \pa{\f{1}{2} J \ba_s + \eta \bX} - \eta \f{2\pi}{L} m_z d}} \overline{\widehat{\vp}}_{s,\bmm,m_z} \widehat{\pa{u_j}}_{\bmm,m_z} \\
		  &=  \sum_{\bmm, m_z} e^{i\pa{\bmm \ba\ind{M}^* \pa{\f{1}{2} J \ba_s + \bX} - \eta \f{2\pi}{L} m_z d}} \overline{\widehat{\vp}}_{s,\eta \bmm,m_z} \widehat{\pa{u_j}}_{\eta \bmm,m_z}
\end{align*}
has Fourier coefficients
\begin{align*}
\widehat{\pa{F^{\eta,j,s}}}_{\bmm} = e^{i \f{1}{2} \bmm \ba\ind{M}^* \cdot J a_s} \sum_{m_z} e^{-i \eta \f{2\pi}{L} m_z d} \overline{\widehat{\vp}}_{s,\eta \bmm,m_z} \widehat{\pa{u_j}}_{\eta \bmm,m_z}
\end{align*}

On the functions given by DFTK, we remark that $\vp_s[m]$ given is periodic and that 
\begin{align*}
\cR_{\f{2\pi}{3}} \vp_{\text{Bl},s} = \tau^s \vp_{\text{Bl},s}.
\end{align*}




\subsection{Symmetries}%
\label{sub:symmetries}

We have
\begin{align*}
	\cR_{\f{2\pi}{3}} F^{\eta,j,s} &= \int_{\R^3} \overline{\vp_{\text{Bl},s}(\by,z)} \Phi_j \pa{R_{-\f{2\pi}3} \pa{R_{\f{2\pi}{3}}\by + R_{\f{2\pi}{3}} \ba_s -2\eta J \bX},z-\eta d} \d \by \d z \\
 &= \int_{\R^3} \overline{\cR_{\f{2\pi}{3}}\vp_{\text{Bl},s}(\by,z)} \pa{\cR_{\f{2\pi}{3}} \Phi_j} \pa{\by + R_{\f{2\pi}{3}} \ba_s -2\eta J \bX,z-\eta d} \d \by \d z \\
	&= \tau^{j-s} \int_{\R^3} \overline{ \vp_{\text{Bl},s}(\by,z)} \Phi_j \pa{\by + R_{\f{2\pi}{3}} \ba_s -2\eta J \bX,z-\eta d} \d \by \d z \\
\end{align*}
and if $\vp_{\text{Bl},s}(y + R_{\f{2\pi}{3}} a_s) = \vp_{\text{Bl},s}(y + a_s)$, then 
\begin{align*}
\cR_{\f{2\pi}{3}} \pa{ \overline{F^{\eta,j,s}} F^{\eta,j',s}} = \tau^{j'-j} \; \overline{F^{\eta,j,s}} F^{\eta,j',s}
\end{align*}



\section{Change of basis for getting $\Phi_j \in L^2_{\tau,\overline{\tau}}$}%
\label{sub:change_of_basis_for_getting_l_2__tau_tau}
Numerically, DFTK gives 
\begin{align*}
\phi, \p \in \Ker \pa{\cR_{\f{2\pi}{3}}-\tau} + \Ker\pa{\cR_{\f{2\pi}{3}}-\overline{\tau}}
\end{align*}
but we want to separate the spaces and obtain $\phi_1 \in \Ker \pa{\cR_{\f{2\pi}{3}}-\tau}$ so that $\phi_2(x,z) := \overline{\phi_1}(-x,z) \in \Ker\pa{\cR_{\f{2\pi}{3}}-\overline{\tau}}$, which existence is ensured by \cite{FefWei12}.

First we define
\begin{align*}
c := \nor{\pa{\cR_{\f{2\pi}{3}} -\tau}\phi_a}{L^2}^2, \qquad s := \ps{\pa{\cR_{\f{2\pi}{3}} -\tau}\phi_a,\pa{\cR_{\f{2\pi}{3}} -\tau}\phi_b}.
\end{align*}
Then we parametrize
\begin{align*}
\phi_1 = e^{i \alpha} \pa{\f s{\ab{s}} \cos \theta \phi_a + e^{i\beta} \sin \theta \phi_b}
\end{align*}
and we want $\pa{\cR_{\f{2\pi}{3}} -\tau}\phi_1 = 0$ which is equivalent to
\begin{align*}
\f s{\ab{s}}\cos \theta \pa{\cR_{\f{2\pi}{3}} -\tau}\phi_a + e^{i\beta} \sin \theta \pa{\cR_{\f{2\pi}{3}} -\tau}\phi_b =0
\end{align*}
and we take the scalar product with $\pa{\cR_{\f{2\pi}{3}} -\tau}\phi_a$ so that
\begin{align*}
\f c{\ab{s}}\cos \theta  + e^{i\beta} \sin \theta =0
\end{align*}
Now we necessarily have $e^{i\beta} = \pm$ so $\cos \theta = \mp \f {\ab{s}}c \sin \theta$ and finally using $\cos^2 + \sin^2 = 1$,
\begin{align*}
\ab{\cos \theta} = \f 1{\sqrt{1+ \pa{\f c{\ab{s}}}^2}}, \qquad \ab{\sin \theta} = \f 1{\sqrt{1+ \pa{\f {\ab{s}}c}^2}},
\end{align*}
and also choosing $\alpha = 0$ if $\cos \theta \ge 0$ and $\pi$ otherwise, which does not change anything, we have
\begin{align*}
\phi_1 = \f{s}{\ab{s}}\f 1{\sqrt{1+ \pa{\f c{\ab{s}}}^2}} \phi_a \pm \f 1{\sqrt{1+ \pa{\f {\ab{s}}c}^2}} \phi_b
\end{align*}
and $\phi_2(x) = \overline{\phi_1(-x)}$. By multiplying by $e^{-iKx}$, we also obtain
\begin{align*}
\boxed{u_1 = \f{s}{\ab{s}}\f 1{\sqrt{1+ \pa{\f c{\ab{s}}}^2}} u_a \pm \f 1{\sqrt{1+ \pa{\f {\ab{s}}c}^2}} u_b}
\end{align*}
and $u_2(x) = \overline{u_1(-x)}$.


\section{The $1/3$ scaling of coordinates}%
\label{sec:the_1_3_scaling_of_coordinates}

Taken from \cite[Appendix G.3, G.4]{WatLus21} for instance, the moiré lattice vectors are 
\begin{align*}
a_1 = \f{2\pi}{3k_\theta} \mat{\sqrt 3 \\ 1},\qquad a_2 = \f{2\pi}{3k_\theta} \mat{-\sqrt 3 \\ 1}
\end{align*}
and $T(x) = \sum_{j=1}^3 T_j e^{-iq_j x}$ has 
\begin{align*}
q_1 = k_\theta \mat{0 \\ -1},\qquad q_{2,3} = \f{k_\theta}{2} \mat{\pm \sqrt 3 \\ 1}
\end{align*}
and we remark that $a_1 \cdot q_1 = -\f{2\pi}{3}$ so actually $q_j \not\in \L^*$ but $3 q_j \in \L^*$.

\section{Change of gauge on the phasis of wavefunctions}%
\label{sec:change_of_gauge_on_the_phasis_of_wavefunctions}

When we change $\Phi_1 \rightarrow \Phi_1 e^{i\theta}$, then $u_1 \rightarrow u_1 e^{i\theta}$, $u_2 \rightarrow u_2 e^{-i\theta}$ because $u_2(x) = \overline{u_1(-x)}$, and hence
\begin{align*}
\boxed{\overline{u_1} u_2 \rightarrow \overline{u_1} u_2 e^{-2i\theta}}
\end{align*}
We define
\begin{align*}
	\cU := \mat{U & 0 \\ 0 & U}
\end{align*}
have
\begin{align*}
	\mat{U & 0 \\ 0 & U} \mat{\bbW^+ & \bbV \\ \bbV^* & \bbW^-}  \mat{U^* & 0 \\ 0 & U^*} = \mat{U\bbW^+ U^* & U\bbV U^*\\ U\bbV^*U^* & U\bbW^-U^*}
\end{align*}
and with $U := \mat{e^{i\theta} & 0 \\  0 & e^{-i\theta}}$, we have
\begin{align*}
	U \mat{B^+ & B \\ B^* & B^-} U^* = \mat{B^+ & B e^{2i\theta} \\ B^* e^{-2i\theta} & B^-}
\end{align*}
hence if we define $H_\theta$ to be $H$ with $u_1 \rightarrow u_1 e^{i\theta}$, we have that
\begin{align*}
\cU H_\theta \cU^*
\end{align*}
is constant in $\theta$.


\section{Comparision between BM and our model}%
\label{sec:comparision_between_bm_and_our_model}


\subsection{Rescaling}%
\label{sub:rescaling}

The BM Hamiltonian is
\begin{align*}
\mat{-i v_0 \sigma \na & w_1 T^{\text{TKV}}(k_\theta x) \\ w_1 T^{*,\text{TKV}}(k_\theta x) & -iv_0 \sigma \na}.
\end{align*}
We consider the rescaling
\begin{align*}
Su(x) := u\pa{\f{x}{k_\theta}}, \qquad S^*u(y) = k_\theta^2 u\pa{k_\theta y}, \qquad S S^* = k_\theta^2
\end{align*}
where we defined $S^*$ as $\int_\Omega \overline{f} \; Sg = \int_{L\Omega/k_\theta} g \; \overline{S^*f}$.
We have $\na S^* = k_\theta S^* \na$ so $S \na S^* = k_\theta^3 \na$ and $SfS^* = k_\theta^2 f\pa{\f{x}{k_\theta}}$ so when $x = y k_\theta$ is the microscopic scale
\begin{multline*}
	\f{1}{k_\theta^3 v_0} S \pa{\mat{-i v_0 \sigma \na & w_1 T^{\text{TKV}}(k_\theta x) \\ w_1 T^{*,\text{TKV}}(k_\theta x) & -iv_0 \sigma \na} -E} S^* \\
	= \mat{-i\sigma \na & \alpha T^{\text{TKV}}\pa{x} \\ \alpha T^{*,\text{TKV}}(x) & -i\sigma \na} - \ep
\end{multline*}
where $\alpha := \f{w_1}{k_\theta v_0}$ and where $\ep = \f{E}{v_0 k_\theta}$ is the unit of \cite[Fig 1]{TarKruVis19} defined in the caption, and
\begin{align*}
T^{\text{TKV}}(x) = \sum_{j=1}^3 T_j e^{-i q_j x}, \qquad q_{2,3} = \mat{\pm \sqrt{3}/2 \\ 1/2}, \qquad q_1 = - q_2 - q_3.
\end{align*}%Hence we need to multiply our energies by $v_0 k_\theta \ll 1$ to obtain the true ones. 

% We will do another rescaling because in our numerical simulations we do not implement $m_{q_j}$ but a transformation of it.
% \begin{align*}
% G u(x) := u\pa{Lx}, \qquad G ^*u(y) = \f{1}{\ab{\det L}} u\pa{ L^{-1}y}, \qquad G  G ^* = \f{1}{\ab{\det L}}
% \end{align*}
% where we defined $G ^*$ as $\int_\Omega \overline{f} \; G g = \int_{L\Omega} g \; \overline{G ^*f}$, and the multiplication operator $T$ transforms as
% \begin{align*}
% G  T G ^* = \f{1}{\ab{\det L}} T\pa{Lx}, \qquad G  \na G ^* = \f{1}{\ab{\det L}} \pa{L^{-1}}^* \na
% \end{align*}
% and
% \begin{align*}
	% T\pa{Lx} &= \sum T_j e^{i m_{q_j} a^* \cdot Lx} = \sum T_j e^{i\pa{\widetilde{L} m_{q_j}} a^* x} = \sum T_j e^{i m_{Q_j} a^* x} \\
% &=: \widetilde{T}(x)
% \end{align*}
% where $\widetilde{L} = \cM^* L^* \pa{\cM^*}^{-1}$. From the numerical needs, we want
% \begin{align*}
% \widetilde{L} = \mat{-1 & 2 \\ -2 & 1}
% \end{align*}
% and hence
% \begin{align*}
% L = \cM \pa{\widetilde{L}}^* \cM^{-1} = -\sqrt 3  J, \qquad \pa{L^{-1}}^* = -\f{1}{\sqrt{3}} J
% \end{align*}
% \begin{align*}
% \ab{\det L} G \mat{-i\sigma \na - \ep & \alpha T\pa{x} \\ \alpha T^*(x) & -i\sigma \na - \ep}  G ^* = \mat{ i\f{1}{\sqrt 3}\sigma  J \na - \ep & \alpha \widetilde{T}\pa{x} \\ \alpha \widetilde{T}^*(x) &  i\f{1}{\sqrt 3}\sigma J \na - \ep}
% \end{align*}


\subsection{Rotation and reduced coordinates of $q$}%
\label{sub:rotation_and_reduced_coordinates_of_q_}


In \cite{TarKruVis19}, the orientation of the lattice (one of the equations below (6)), is with reciprocal vectors
\begin{align*}
b^{*,\text{TKV}}_{1,2} = \sqrt 3 \mat{\pm 1/2 \\ \sqrt{3}/2}
\end{align*}
and to compare with our lattice defined in \eqref{def:lattice_b}, we have
\begin{align*}
-J b_1^{*,\text{TKV}} = b_1^*, \qquad -J b_2^{*,\text{TKV}} = -b_2^*, \qquad b = \f{4\pi}{3}
\end{align*}
corresponding to the direct lattice 
\begin{align*}
b_1 = b \mat{-1/2 \\ \sqrt{3} /2}, \qquad  b_2 = b\mat{1/2 \\ \sqrt{3}/2}, \qquad \cM_b = \f{b}{2} \mat{-1 & 1 \\ \sqrt{3} & \sqrt{3}}
\end{align*}
The $q_j$'s are
\begin{align*}
q^{\text{TKV}}_{2,3} = \mat{\pm\f{\sqrt 3}{2} \\ \f{1}{2}}, \qquad q^{\text{TKV}}_1 = -q^{\text{TKV}}_2 - q^{\text{TKV}}_3
\end{align*}
so
\begin{align*}
\sqrt{3} J q_{2,3}^{\text{TKV}} = \pm b_{2,1}^{*,\text{TKV}}
\end{align*}
and we do a rotation, $q_j := -J q^{\text{TKV}}_j$,
\begin{align*}
q_1 = \mat{-1 \\ 0}, \qquad q_{2,3} = \mat{\f 12 \\ \mp \f{\sqrt 3}{2} }, \qquad  q_1 = -q_2 - q_3
\end{align*}
We have
\begin{align*}
T(x) := T^{\text{TKV}}(Jx) = \sum_j T_j e^{-iq_j x} \underset{\substack{\eqref{eq:inv_q}}}{=} \; \sum_j T_j e^{i\widetilde{m}_{q_j} b^* x}
\end{align*}
where $\widetilde{m}_{q_j} = -\f{1}{2\pi} \cM_b^* q_j$, that is
\begin{align*}
\widetilde{m}_{q_1} = \f{1}{3} \mat{-1 \\ 1}, \qquad \widetilde{m}_{q_2} = \f{1}{3} \mat{2 \\ 1}, \qquad \widetilde{m}_{q_3} =  \f{1}{3} \mat{-1 \\ -2}
\end{align*}
and we redefine
\begin{align}\label{eq:fourier_TBM}
m_{q_1} =  \mat{-1 \\ 1}, \qquad m_{q_2} =  \mat{2 \\ 1}, \qquad m_{q_3} =   \mat{-1 \\ -2}
\end{align}
so
\begin{align}\label{eq:bm_our_way}
T(x) = \sum_j T_j e^{i m_{q_j} \f{b^*}{3} x}
\end{align}
We conjugate again and get
\begin{multline*}
\cR_{-\f{\pi}{2}} \pa{ \mat{-i\sigma \na & \alpha T^{\text{TKV}}\pa{x} \\ \alpha T^{*,\text{TKV}}(x) & -i\sigma \na} - \ep}\cR_{\f{\pi}{2}} \\
= \mat{-i\sigma \cdot J\na & \alpha T\pa{x} \\ \alpha T^{*}(x) & -i\sigma \cdot J\na} - \ep
\end{multline*}
the action of $J$ corresponding to the rotation of the dual lattice vectors, so if we write $\na$ in our new lattice $b$, we have
\begin{align*}
\mat{-i\sigma \na & \alpha T\pa{x} \\ \alpha T^{*}(x) & -i\sigma\na} - \ep
\end{align*}

\subsection{Rescaling again}%
\label{sub:rescaling_again}

To write the Fourier coefficients of $T$, we need to rescale, so ne define $Su(x) := u(3x)$ and as previously, doing ``$k_\theta = 1/3$'', we have $S S^* = 1/9$, $S \na S^* = \pa{1/3^3} \na$
\begin{align*}
3^2 S \pa{\mat{-i\sigma \na & \alpha T\pa{x} \\ \alpha T^{*}(x) & -i\sigma\na} - \ep} S^* = \mat{-\f{1}{3}i\sigma \na & \alpha T\pa{3x} \\ \alpha T^{*}(3x) & - \f{1}{3} i\sigma\na} - \ep
\end{align*}
and now we can implement the Fourier coefficients of $T(3\cdot)$, given by \eqref{eq:fourier_TBM}, because
\begin{align*}
T(3x) = \sum_j T_j e^{i m_{q_j} b^* x}
\end{align*}



\subsection{Relation to our model}%
\label{sub:relation_to_our_model}



We compute, for $j \in \acs{1,2,3}$,
\begin{align*}
	v^m_\theta(x) &= v_m e^{ima^* \pa{\cos \f{\theta}{2} x + \sin \f{\theta}{2} J x}} +   v_m e^{ima^* \pa{\cos \f{\theta}{2} x - \sin \f{\theta}{2} J x}} \\
		    &= 2i v_m e^{ima^* \cos \f{\theta}{2} x}  \sin ma^* \sin \f{\theta}{2}  Jx \\
		    &=  2i v_m e^{ima^* \cos \f{\theta}{2} x}  \sin m\f{a^*}{2 k_D} k_\theta J x
\end{align*}
and
\begin{align*}
\f{a^*_1}{2 k_D} = \f{\sqrt 3}{2} \mat{ \f{1}{2} \\ \f{\sqrt 3}{2}} = - \f{\sqrt 3}{2} Jq_2, \qquad \f{a^*_2}{2 k_D} = \f{\sqrt 3}{2} \mat{ \f{1}{2} \\ -\f{\sqrt 3}{2}}= - \f{\sqrt 3}{2} Jq_3
\end{align*}
We deduce that
\begin{align*}
	v^m_\theta(x) =  -2i v_m e^{ima^* \cos \f{\theta}{2} x}  \sin k_\theta \f{\sqrt 3}{2} mb^* \cdot x
\end{align*}
where $b^*_1 := q_2$, $b^*_2 := q_3$. We define $m_2=(1,0)$, $m_3=(0,1)$, $m_1 = (-1,-1)$, so the three modes are
\begin{align*}
m_j b^* = q_j
\end{align*}
and
\begin{align*}
	v^{m_j}_\theta(x) &=  \cdot \sin k_\theta \f{\sqrt 3}{2} q_j \cdot x 
\end{align*}



\section{Code}%
\label{sec:code}

\begin{multline*}
b_1^* = \f{4\pi \ep}{a\sqrt{3}} \mat{-1/2 \\ -\sqrt{3}/2} = k_\theta \sqrt{3} \mat{-1/2 \\ -\sqrt{3}/2},\quad b_2^* = k_\theta \sqrt{3} \mat{1 \\ 0} \\
R_{\f{2\pi}{3}} b_1^*= b_2^*,\quad R_{\f{2\pi}{3}} b_2^* = -b_1^* - b_2^* \\
K_1 = \f 13 \pa{b_1^* + 2 b_2^*} = \f 13 \pa{a_1^* -2a_2^*} = k_\theta \mat{\sqrt{3}/2 \\ -1/2} \\
K_2 = \f 13 \pa{-b_1^* +  b_2^*}= \f 13 \pa{2a_1^*  -a_2^*} = k_\theta \mat{\sqrt{3}/2 \\ 1/2} = R_{\f{2\pi}{6}} K_1
\end{multline*}
then
\begin{align*}
	H = \mat{\sigma^\xi \cdot \pa{-i\na + k - K_1} & T_\xi(x) \\ T_\xi(x)^* & \sigma^\xi \cdot \pa{-i\na + k - K_2}}
\end{align*}
where $\sigma^\xi := \pa{\xi \sigma_1,\sigma_2}$ and $T_\xi(x) = T^\xi_1  + T^\xi_2 e^{-i b_1^* x} + T^\xi_3 e^{-i(b_1^* + b_2^*) x}$ where
\begin{align*}
	T_1^\xi = \mat{\alpha & \beta \\ \beta & \alpha},\qquad T_2^\xi = \mat{\alpha & \beta e^{-i\xi\phi} \\ \beta e^{i\xi \phi} & \alpha}\qquad T^\xi_3 = \overline{T_2^\xi}
\end{align*}
and with $U := \mat{e^{iK_1 x} & 0 \\ 0 & e^{iK_2 x}}$, we have
\begin{align*}
U^* H U = \mat{\sigma^\xi \cdot \pa{-i\na + k} & T_\xi(x) e^{i\pa{K_1 - K_2}x} \\ T_\xi(x)^*e^{i\pa{K_2 - K_1}x} & \sigma^\xi \cdot \pa{-i\na + k}}
\end{align*}
and $K_1 - K_2 = \f 13 \pa{2b_1^* + b_2^*}$ so
\begin{align*}
T_\xi(x) e^{i\pa{K_1 - K_2}x} = T_1 e^{i \f 13 \pa{2b_1^* + b_2^*}} + T_2 e^{i \f 13 \pa{-b_1^* + b_2^*}}+ T_3 e^{i \f 13 \pa{-b_1^*-2 b_2^*} }
\end{align*}
If we define
\begin{multline*}
Q_1 := \f 13 \pa{2b_1^* + b_2^*} = k_\theta \mat{0 \\ -1} = q_1 \\
Q_2 := \f 13\pa{-b_1^* + b_2^*},\qquad Q_3 :=  \f 13 \pa{-b_1^* - 2b_2^*}
\end{multline*}
we have
\begin{align*}
R_{\f{2\pi}{3}} Q_1 = Q_2,\qquad R_{\f{2\pi}{3}} Q_2 = Q_3,\qquad R_{\f{2\pi}{3}} Q_3 = Q_1
\end{align*}
and hence $Q_j = q_j$, of TKV, and $TT_\xi(x) e^{i\pa{K_1 -K_2}x} = T_\xi(x) e^{iq_1 x} = T(x)$. Also, we have $b_1^* + b_2^* = k_\theta \sqrt{3} \mat{1/2 \\ -\sqrt{3}/2} = -a_2^*$ and $b_1^* = - a_1^*$, so
\begin{align*}
b_1^* = -a_1^*,\qquad b_2^* = a_1^* - a_2^*
\end{align*}




\subsection{TKV}%
\label{sub:tkv}

In TKV, we have
\begin{multline*}
	a_1^* = q_2 - q_1 = k_\theta \sqrt{3} \mat{1/2 \\ \sqrt{3}/2},\qquad a_2^* = q_3 - q_1 = k_\theta \sqrt{3} \mat{-1/2 \\ \sqrt{3}/2} \\
	q_1 = -q_2 - q_3
\end{multline*}
To have $q_j$ in terms of $a_j^*$, we compute $a_1^* \pm a_2^*$ and $a_1^* = 2q_2 + q_3$, $a_2^* = q_2 + 2 q_3$ and
\begin{align*}
q_3 = \f 13 \pa{-a_1^* + 2 a_2^*},\qquad q_2 = \f 13 \pa{ 2 a_1^* - a_2^*},\qquad q_1 = \f 13 \pa{-a_1^* - a_2^*} \\
R_{\f{2\pi}{3}} q_1 = q_2,\qquad R_{\f{2\pi}{3}} q_2 = q_3,\qquad R_{\f{2\pi}{3}} q_3 = q_1
\end{align*}
(this was triples checked, including with the cartesian coordinates). Moreover,
\begin{multline*}
R_{-\f{2\pi}{6}} a_1^* = a_1^* - a_2^*,\qquad R_{-\f{2\pi}{6}} a_2^* = a_1^* \\
R_{\f{2\pi}{6}} a_1^* = a_2^*,\qquad R_{\f{2\pi}{6}} a_2^* = a_2^* - a_1^*
\end{multline*}
If 
\begin{align*}
	S := \mat{a_1^* & a_2^*} = k_\theta \sqrt{3} \mat{1/2 & -1/2 \\ \sqrt{3}/2 & \sqrt{3}/2}, \qquad S^{-1} = \f{2}{3 k_\theta} \mat{\sqrt{3}/2 & 1/2 \\ -\sqrt{3}/2 & 1/2}
\end{align*}
and
\begin{align*}
\cM = \mat{a_1 & a_2} = 2\pi \pa{S^*}^{-1} = \f{4\pi}{3k_\theta} \mat{\sqrt{3}/2 & - \sqrt{3}/2 \\ 1/2 & 1/2}
\end{align*}

with $U := \mat{e^{-iK_1 x} & 0 \\ 0 & e^{-iK_2 x}}$, and
\begin{align*}
T(x) = T_1 e^{-iq_1 x}  +T_2 e^{-iq_2 x}  +T_3 e^{-iq_3 x} 
\end{align*}
we compute
\begin{align}\label{eq:UmatU}
	U^* \mat{-i\sigma \na & T \\ T^* & -i\sigma \na} U = \mat{\sigma\pa{-i\na - K_1} & T e^{i\pa{K_1 - K_2}x} \\ T^* e^{i\pa{K_2-K_1}x} & \sigma\pa{-i\na - K_2}} 
\end{align}
and with $K_1 - K_2 = q_1$,
\begin{align*}
T e^{i\pa{K_1 - K_2}x} = T_1 + T_2 e^{i\pa{-q_2+q_1}x} + T_3 e^{i\pa{-q_3+q_1}x} = T_1 + T_2 e^{-i a_1^* x} + T_3 e^{-ia_2^* x}
\end{align*}
% moreover,
% \begin{align*}
	% R^\text{red}_{-\f{2\pi}{6}} = S^{-1} R_{-\f{2\pi}{6}} S = \mat{1 & 1 \\ -1 & 0}
% \end{align*}
 and if $K_1 = \f 13 \pa{\alpha a_1^* + \beta a_2^*}$, then 
\begin{align*}
	K_2 &:= R_{-\f{2\pi}{6}} K_1 = \f 13 \pa{\alpha a_1^* + \pa{\beta-\alpha}a_2^*} \\
K_1 - K_2 &= \f 13 \alpha a_2^* = q_1 = \f 13 \pa{-a_1^* - a_2 ^*}
\end{align*}
has no solution !!! We can try
\begin{align*}
	K_2 &:= R_{\f{2\pi}{6}} K_1 = \f 13 \pa{-\beta a_1^* + \pa{\alpha + \beta}a_2^*} \\
K_1 - K_2 &= \f 13 \pa{\pa{\alpha -\beta}a_1^* - \alpha a_2^*} = q_1 = \f 13 \pa{-a_1^* - a_2 ^*}
\end{align*}
so $(\alpha,\beta) = (1,2)$ but then $K_1$ is not a Dirac point for this configuration !

\subsection{Deductive from TKV}%
\label{sub:deductive_from_tkv}

From \eqref{eq:UmatU}, that we consider again, we want $K_1- K_2 = q_1$, so 
\begin{align*}
	T(x) e^{i(K_1-K_2)x} &= T(x) e^{iq_1 x} = T_1 +T_2 e^{i(q_1 - q_2)x} +T_3 e^{i(q_1 - q_3)x} \\
& = T_1 +T_2 e^{-ic_1^*x} +T_3 e^{-i\pa{c_1^*+c_2^*}x}
\end{align*}
where
\begin{multline*}
c_1^* := q_2 - q_1 = a_1^*, \qquad c_2^* := q_3 - q_2 = -a_1^* + a_2^*, \qquad a_2^* = c_1^* + c_2^*
\end{multline*}
hence
\begin{align*}
q_1 = \f 13\pa{-2c_1^* - c_2^*}, \qquad q_2 = \f 13\pa{c_1^* - c_2^*},\qquad q_3 = \f 13\pa{c_1^* +2 c_2^*}
\end{align*}
and
\begin{align*}
R_{\f{2\pi}{6}} c_1^* = c_1^* + c_2^*, \qquad R_{\f{2\pi}{6}} c_2^* = -c_1^*
\end{align*}
so if $K_2 = \f 13 \pa{\alpha c_1^* + \beta c_2^*}$, and if we impose $K_1 = R_{\f{2\pi}{6}} K_2$, we have $K_1 = \f 13 \pa{\pa{\alpha - \beta} c_1^* + \alpha c_2^*}$ and
\begin{align*}
\f 13 \pa{-2c_1^* -c_2^*} = q_1 = K_1 - K_2 = \f 13\pa{-\beta c_1^* + \pa{\alpha - \beta} c_2^*} \implies (\alpha,\beta) = (1,2)
\end{align*}
so
\begin{align*}
K_1 = \f 13 \pa{ -c_1^* + c_2^*},\qquad K_2 = \f 13 \pa{c_1^* + 2c_2^*}
\end{align*}

\bibliographystyle{siam}
\bibliography{../../../biblio}
\end{document}
