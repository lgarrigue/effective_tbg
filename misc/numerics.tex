\documentclass[11pt,a4paper,reqno,french,tikz]{amsart}
\usepackage[utf8]{inputenc} \usepackage[T1]{fontenc}
%\usepackage{babel} % français \usepackage[latin1]{inputenc}
\usepackage{amsthm,amsmath,amsfonts,amssymb,amsxtra,appendix,bookmark,dsfont,bm,mathrsfs,amstext,amsopn,mathrsfs,mathtools,comment,cite,hyperref,color,xcolor,cite}

\usepackage{stmaryrd}

%%%%%%%%%%%%%%%%%%%% Standard macros
% Environments
\newtheorem{theorem}{Theorem}[section]\newtheorem{definition}[theorem]{Definition}\newtheorem{lemma}[theorem]{Lemma}\newtheorem{example}[theorem]{Example}\newtheorem{proposition}[theorem]{Proposition}\newtheorem{corollary}[theorem]{Corollary}\newtheorem{conjecture}[theorem]{Conjecture}\newtheorem{remark}[theorem]{Remark}

% Mathbb
\let\C\relax\newcommand{\C}{\mathbb{C}}\newcommand{\Z}{\mathbb{Z}}\newcommand{\R}{\mathbb{R}}\newcommand{\N}{\mathbb{N}}\newcommand{\Q}{\mathbb{Q}}

% Mathcal
\newcommand\cA{\mathcal{A}}\newcommand\cB{\mathcal{B}}\newcommand\cC{\mathcal{C}}\newcommand\cD{\mathcal{D}}\newcommand\cE{\mathcal{E}}\newcommand\cF{\mathcal{F}}\newcommand\cG{\mathcal{G}}\newcommand\cH{\mathcal{H}}\newcommand\cI{\mathcal{I}}\newcommand\cJ{\mathcal{J}}\newcommand\cK{\mathcal{K}}\newcommand\cL{\mathcal{L}}\newcommand\cM{\mathcal{M}}\newcommand\cN{\mathcal{N}}\newcommand\cO{\mathcal{O}}\newcommand\cP{\mathcal{P}}\newcommand\cQ{\mathcal{Q}}\newcommand\cR{\mathcal{R}}\newcommand\cS{\mathcal{S}}\newcommand\cT{\mathcal{T}}\newcommand\cU{\mathcal{U}}\newcommand\cV{\mathcal{V}}\newcommand\cW{\mathcal{W}}\newcommand\cX{\mathcal{X}}\newcommand\cY{\mathcal{Y}}\newcommand\cZ{\mathcal{Z}}

% Math operators
\DeclareMathOperator{\tr}{Tr}\DeclareMathOperator{\ran}{Ran}\DeclareMathOperator{\Span}{Span}\DeclareMathOperator{\Ker}{Ker}\DeclareMathOperator{\re}{Re}\DeclareMathOperator{\id}{id}\DeclareMathOperator{\im}{Im}\DeclareMathOperator{\dist}{dist}\let\div\relax\DeclareMathOperator{\div}{div}
\def\d{{\rm d}}

% Others
\renewcommand{\ge}{\geqslant}\renewcommand{\le}{\leqslant}

\newcommand{\intent}[1]{\llbracket #1 \rrbracket}

%%%%%%%%%%%%%%%%%%%% Macros Louis
% Delimiters
\newcommand{\pa}[1]{\left( #1 \right)} % ()
\newcommand{\acs}[1]{\left\{ #1 \right\}} % {}
\newcommand{\seg}[1]{\left[ #1 \right]} % []
\newcommand{\ab}[1]{\left|#1\right|} % ||
\newcommand{\ps}[1]{\left< #1 \right>} % <>
\newcommand{\proj}[1]{\big| #1 \big> \big< #1 \big|} % |u><u|
\newcommand{\nor}[2]{ \left| \! \left| #1 \right| \! \right|_{#2} } % ||.||

% Greek letters shortcuts
\newcommand\vp{\varphi} % φ
\def\eps{\varepsilon}\newcommand{\ep}{\varepsilon} % ε
\let\p\relax\newcommand{\p}{\psi} % ψ
\newcommand{\na}{\nabla} % ∇

% Others
\newcommand{\f}[2]{\frac{#1}{#2}} % fraction
\newcommand{\bul}{$\bullet$ \hspace{0.1cm}} % •
\newcommand{\mymax}[1]{\underset{\substack{#1}}{\text{\normalfont{max}}}\quad} % max
\newcommand{\mymin}[1]{\underset{\substack{#1}}{\text{\normalfont{min}}}\quad} % min
\newcommand{\mysup}[1]{\underset{\substack{#1}}{\text{\normalfont{sup}}}\quad} % sup
\newcommand{\myinf}[1]{\underset{\substack{#1}}{\text{\normalfont{inf}}}\quad} % inf
\newcommand{\ind}[1]{_{\textup{#1}}} % indice
\newcommand{\apo}[1]{#1''} % apostrophes "..."
\newcommand{\mat}[1]{\begin{pmatrix} #1 \end{pmatrix}} % matrices
\def\1{{\mathds{1}}}
\newcommand{\ketbra}[2]{\left| #1 \right> \left< #2 \right|}
\newcommand{\bbV}{\mathbb{V}}


\newcommand{\db}[1]{\left(\!\left( #1 \right)\!\right)}
\def\bX{{\mathbf X}}
\def\bG{{\mathbf G}}
\def\ba{{\mathbf a}}
\def\bb{{\mathbf b}}
\def\be{{\mathbf e}}
\def\bk{{\mathbf k}}
\def\bp{{\mathbf p}}
\def\bq{{\mathbf q}}
\def\br{{\mathbf r}}
\def\bx{{\mathbf x}}
\def\bmm{{\mathbf m}}
\def\bv{{\mathbf v}}
\def\bs{{\mathbf s}}
\def\by{{\mathbf y}}
\def\bn{{\mathbf n}}
\def\bA{{\mathbf A}}
\def\bH{{\mathbf H}}
\def\bK{{\mathbf K}}
\def\bR{{\mathbf R}}
\def\bS{{\mathbf S}}
\def\gR{{\mathfrak R}}

\def\bbI{{\mathbb I}}
\def\bbV{{\mathbb V}}
\def\bbW{{\mathbb W}}
\def\bbA{{\mathbb A}}
\def\bbB{{\mathbb B}}
\def\1{{\mathds{1}}}

\newcommand{\dd}{\tfrac{d}{2}}

\def\lAngle{\langle\!\langle}
\def\rAngle{\rangle\!\rangle}

%%%%%%%%%%%%%%%%%%%% End macros

\title[Numerics effective TBG]{Numerical computations for an effective model of twisted bilayer graphene}
\author[É. Cancès, L. Garrigue and D. Gontier]{Éric Cancès, Louis Garrigue and David Gontier}
% \address{CERMICS, \'Ecole des ponts ParisTech, 6 and 8 av. Pascal, 77455 Marne-la-Vallée, France} 
% \email{louis.garrigue@enpc.fr}
% \date{\today}
\begin{document}
\maketitle

\section{Standard monolayer}%
\label{sec:standard_monolayer}

We recall that 
\begin{multline*}
a_1 = a \mat{ \f{\sqrt 3}2 \\ \f 12}, \qquad a_2 = a \mat{ \f{\sqrt 3}2 \\ -\f 12} \\
a^*_1 = \f{2\pi}{a} \mat{ \f{1}{\sqrt 3} \\ 1}= \f{4\pi}{a\sqrt{3}} \mat{ \f{1}{2} \\ \f{\sqrt{3}}{2}}, \qquad a^*_2 = \f{2\pi}{a} \mat{ \f{1}{\sqrt 3} \\ -1}= \f{4\pi}{a\sqrt{3}} \mat{ \f{1}{2} \\ -\f{\sqrt{3}}{2}}
\end{multline*}

In reduced coordinates, with 
\begin{multline*}
	\cM : \mathbb{T}^2 \simeq [0,1]^2 \rightarrow \Omega, \\
	\cM :=  \f a2 \mat{\sqrt{3} & \sqrt{3} \\ 1 & -1} = \mat{a_1 & a_2}, \qquad  \cM^{-1} = \f 1a \mat{\f 1{\sqrt 3} & 1 \\ \f 1{\sqrt 3} & -1}
\end{multline*}
and
\begin{align*}
	2\pi \pa{\cM^{-1}}^* = \mat{a_1^* & a_2^*} = \f{2\pi}a \mat{\f{1}{\sqrt 3} & \f{1}{\sqrt 3} \\ 1 & -1} =: S
\end{align*}

\subsection{Fourier conventions}%
\label{sub:fourier_conventions}

We will manipulate functions which are $\Omega$-periodic in $\bx$, but not in $z$, our Fourier transform conventions will be
\begin{align*}
	\pa{\cF f}_G(k_z) := \f{1}{2\pi \ab{\Omega}} \int_{\Omega\times\R} e^{-i\pa{k \bx + k_z z}} f(\bx,z) \d \bx \d z
\end{align*}
hence any function can be decomposed as
\begin{align*}
f(\bx,z) = \sum_{\bG \in \L^*} \int_\R e^{i\pa{\bG\bx + k_z z}} f_\bG(k_z) \d k_z
\end{align*}
We also recall that $\int_\R e^{ipz} \d z = 2\pi \delta(p)$.


Now we consider that $f$ and $g$ are $L$-periodic in $z$,
\begin{align*}
f(\bx,z) = \sum_{\bG,G_z} e^{i\pa{\bG \bx + G_z z}} \widehat{f}_{\bG,G_z}
\end{align*}
or, in reduced coordinates,
\begin{align}\label{eq:dec_four}
\boxed{f(\bx,z) =  \sum_{\substack{\bmm \in \Z^2 \\ m_z \in \Z}}  e^{i\pa{\bmm \ba^* \cdot \bx + m_z \f{2\pi}L z}} \widehat{f}_{\bmm,m_z}}
\end{align}
We define the scalar product
\begin{align*}
\ps{f,g} := \int_{\Omega\times [0,L]} \overline{f}g
\end{align*}
and compute Plancherel's formula
\begin{align}\label{eq:plancherel}
\ps{f,g} = L\ab{\Omega} \sum_{\substack{\bmm \in \Z^2 \\ m_z \in \Z}} \overline{\widehat{f}_{\bmm,m_z}} \widehat{g}_{\bmm,m_z}.
\end{align}
Hence, as a verification, we test that the normalization of the $\widehat{u_j}$'s is the right one by checking that $\nor{u_j}{L^2}^2 = 1$ via \eqref{eq:plancherel}.

\subsection{Rotation action}%
\label{sub:rotation_action}


We know that $R_{\f{2\pi}3} \pa{ma^* } = \pa{R_{\f{2\pi}3}^{\text{red}} m} a^*$ where
\begin{align*}
	R_{\f{2\pi}3}^{\text{red}} = S^{-1} R_{\f{2\pi}3} S =  \cM^* R_{\f{2\pi}3} \pa{\cM^*}^{-1} = \mat{-1 & 1 \\ -1 & 0}, \qquad R_{-\f{2\pi}3}^{\text{red}} = \mat{0 & -1 \\ 1 & -1}
\end{align*}
and
\begin{align*}
\cR_{\f{2\pi}3} f(x) = \sum_m f_m e^{i \pa{R_{\f{2\pi}3}^{\text{red}} m} a^* \cdot x} = \sum_m f_{R_{-\f{2\pi}3}^{\text{red}} m} e^{im a^* \cdot x}
\end{align*}

\section{Computation of $V\ind{int}$}%
\label{sec:computation_of_vint_}

For $\bs \in \Omega := [0,1] \ba_1 + [0,1] \ba_2$, we denote by $V^{(2)}_{\bs}$ the true Kohn-Sham mean-field potential for the configuration where the two sheets are aligned (no angle), but with the upper one shifted by a vector $\bs$. We set
\begin{align*}
	V_{\rm int, \bs}(z) &:= \f{1}{\ab{\Omega}} \int_{\Omega}  \left( V^{(2)}_{\bs}(\bx, z) - V(\bx, z + \dd) - V(\bx - \bs, z -\dd)   \right) \d \bx \\
    &= \f{1}{\ab{\Omega}} \sum_{\substack{\bmm \in \Z^2 \\ m_z \in \Z}} \int_{\Omega} e^{i \pa{\bmm \ba^* \cdot \bx + m_z \f{2\pi}{L} z}} \\
    &\quad \times \pa{ \widehat{\pa{V^{(2)}_{\bs}}}_{\bmm,m_z} - \widehat{V}_{\bmm,m_z}e^{i m_z \f{2\pi}{L} \f{d}{2}} -\widehat{V}_{\bmm,m_z} e^{-i \pa{\bmm \ba^*\cdot \bs +m_z \f{2\pi}{L} \f{d}{2}}}} \d \bx \\
    &= \sum_{\substack{m_z \in \Z}} e^{i  m_z \f{2\pi}{L} z} \pa{ \widehat{\pa{V^{(2)}_{\bs}}}_{0,m_z} - 2 \widehat{V}_{0,m_z} \cos\pa{ m_z \tfrac{\pi d}{L} }}
\end{align*}
and we obtain the Fourier coefficients
\begin{align*}
\pa{\widehat{V_{\rm{int},\bs}}}_{m_z} = \widehat{\pa{V^{(2)}_{\bs}}}_{0,m_z} - 2 \widehat{V}_{0,m_z} \cos\pa{ m_z \tfrac{\pi d}{L} }
\end{align*}
We then compute
\begin{align*}
V\ind{int}(z) := \f{1}{\ab{\Omega}} \int_\Omega V_{\rm int, \bs}(z) \d \bs = \f{1}{N^2} \sum_{\substack{s_x,s_y \in \intent{1,N}}} V^{\rm{array}}_{\rm int, (s_x,s_y)}(z)
\end{align*}
and finally obtain the Fourier coefficients
\begin{align*}
\boxed{\pa{\widehat{V\ind{int}}}_{m_z} = \f{1}{N^2} \sum_{\substack{s_x,s_y \in \intent{1,N}}} \pa{\widehat{V_{\rm{int},\bs}}}_{m_z}}
\end{align*}
and we expect $V_{\rm{int},\bs}$ not to depend too much on $\bs$, that is we expect that
\begin{align*}
	\delta_{V\ind{int}} &:= \f{\int_{\Omega\times \R} \ab{V_{\rm{int},\bs}(z) - V\ind{int}(z)}^2 \d \bs \d z}{\ab{\Omega} \int_{\R} V\ind{int}(z)^2 \d z} \\
&= \f{\sum_{m_z} \int_{\Omega} \ab{\pa{\widehat{V_{\rm{int},\bs}}}_{m_z} - \pa{\widehat{V\ind{int}}}_{m_z}}^2 \d \bs }{\ab{\Omega} \sum_{m_z}  \pa{\widehat{V\ind{int}}}_{m_z}^2} \\
&= \f{\sum_{s_x,s_y,m_z} \ab{\pa{\widehat{V}_{\rm{int},(s_x,s_y)}}_{m_z} - \pa{\widehat{V\ind{int}}}_{m_z}}^2 }{N^2 \sum_{m_z}  \pa{\widehat{V\ind{int}}}_{m_z}^2}
\end{align*}
is small. We also verify that $V\ind{int}(-z) = V\ind{int}(z)$.



\section{Effective potentials}%
\label{sec:effective_potentials}

We defined
\begin{align*}
\db{ f, g}^{\eta,\eta'}(\bX) :=  \f{1}{\ab{\Omega}} \int_{\Omega \times \R} \overline{f}\pa{x -\eta J\bX,z- \eta\dd} g\pa{x - \eta' J \bX, z- \eta'\dd} \d \bx \d z
\end{align*}
and
\begin{multline*}
\lAngle f, g \rAngle^{\eta,\eta'}(\bX)\\
:=  \f{e^{i\pa{\eta-\eta'} \bK \cdot J \bX}}{\ab{\Omega}} \int_{\Omega \times \R} \overline{f}\pa{x -\eta J\bX,z-\eta \dd} g\pa{x -\eta' J \bX, z-\eta'\dd} \d \bx \d z
\end{multline*}
so $\lAngle f, g \rAngle^{\eta,\eta'} = e^{i\pa{\eta-\eta'} \bK \cdot J \bX}\db{ f, g}^{\eta,\eta'}$. Now we make the approximation
\begin{align*}
\int_{\Omega\times\R} \simeq  \int_{\Omega\times \seg{0,L}}
\end{align*}
The situation is drawn on Figure \ref{fig:z_drawing}. The functions are defined on $[-L/2,L/2]$ but we need to integrate on the common segment, which is $[-\f{L-d}{2},\f{L-d}{2}]$, so on $[-L/2,L/2]$ to recover the initial domain.

\begin{figure}
\begin{center}
\includegraphics[width=10cm,trim={0cm 0cm 0 10cm},clip]{common.jpeg}
\label{fig:z_drawing}\caption{Situation on the $z$ coordinate}
\end{center}
\end{figure}


Firstly, using the Fourier decomposition \eqref{eq:dec_four},
\begin{align*}
	\db{f,g}^{\eta,\eta'} &=  L \sum_{\bmm \in \Z^2} e^{i\pa{\eta-\eta'} ma^* \cdot J \bX} \sum_{m_z \in \Z} e^{i\pa{\eta-\eta'} \f{2\pi}{L} m_z \f{d}{2}} \overline{\widehat{f}_{m,m_z}} \widehat{g}_{m,m_z}\\
& = \sum_{\bmm \in \Z^2} e^{i\pa{\eta-\eta'} ma^* \cdot J \bX} C_\bmm
\end{align*}
where
\begin{align*}
\boxed{C_\bmm := L \sum_{m_z \in \Z} e^{i\pa{\eta-\eta'} \f{d\pi}{L} m_z} \overline{\widehat{f}_{m,m_z}} \widehat{g}_{m,m_z}}
\end{align*}
and we also define
\begin{align*}
C^\pm_\bmm := L \sum_{m_z \in \Z} e^{\pm i2 \f{d\pi}{L} m_z} \overline{\widehat{f}_{m,m_z}} \widehat{g}_{m,m_z}
\end{align*}
Then,
\begin{align*}
\lAngle f,g \rAngle ^{\eta,\eta'} = e^{i\pa{\eta-\eta'} \bK \cdot J \bX}\db{f,g}^{\eta,\eta'} =  \sum_{\bmm \in \Z^2} e^{i\pa{\eta-\eta'} \pa{m+m_K}a^* \cdot J \bX} C_\bmm
\end{align*}


Hence
\begin{align*}
\boxed{\db{f,g}^{+-}\pa{-\tfrac{3}{2} J \bX} =  \sum_{\bmm \in \Z^2} e^{i 3 ma^* \cdot \bX} C^+_\bmm=  \sum_{\bmm \in \Z^2} e^{i m a^* \cdot \bX} C^+_{\f{\bmm}{3}},}
\end{align*}
and
\begin{align*}
\boxed{\lAngle f,g \rAngle^{+-}\pa{-\tfrac{3}{2} J\bX} =  \sum_{\bmm \in \Z^2} e^{i 3 \pa{m+m_k}a^* \cdot \bX} C^+_\bmm =  \sum_{\bmm \in \Z^2} e^{i m a^* \cdot \bX} C^+_{\f{\bmm-3\bmm_K}{3}}}
\end{align*}
where $C_{\f{\bmm}{n}} := 0$ if $n$ does not divide $m_1$ and $m_2$. 

Similarly
\begin{align*}
\db{f,g}^{-+}\pa{-\tfrac{3}{2} J \bX} =  \sum_{\bmm \in \Z^2} e^{-i 3 ma^* \cdot \bX} C^-_\bmm=  \sum_{\bmm \in \Z^2} e^{i m a^* \cdot \bX} C^-_{-\f{\bmm}{3}},
\end{align*}

For the potentials, we finally need to implement
\begin{multline*}
\bbW_{j,j'}^+ = \db{\overline{u}_j u_{j'}, V}^{+-}, \qquad \bbW_{j,j'}^- = \db{\overline{u}_j u_{j'}, V}^{-+},\\
\bbV_{j,j'} = \lAngle \pa{V+V\ind{int}} u_j , u_{j'} \rAngle^{+-}
\end{multline*}


If $f(z) = \ep f(-z)$, then $\widehat{f}_{-m_z} = \ep \widehat{f}_{m_z}$, from this we see that $\overline{C_\bmm^{u_{j'},u_{j}}} = C_\bmm^{u_{j},u_{j'}}$ and hence $\bbV(-X)^* = \bbV(X)$

\subsection{Magnetic term}%
\label{sub:magnetic_term}


As for the magnetic term, we have
\begin{align*}
\pa{-i\na_\bx + \bK} g = \sum_{\bmm,m_z} \pa{\bmm + \bmm_K} \ba^* e^{i\pa{\bmm \ba^* \cdot \bx + m_z \f{2\pi}L z}} \widehat{f}_{\bmm,m_z}
\end{align*}
so
\begin{align*}
	\lAngle &f, (-i\nabla_\bx+\bK) g \rAngle^{+-}(\bX) =  \sum_{\bmm \in \Z^2} \pa{\bmm + \bmm_K}\ba^* \; C_\bmm e^{2i \pa{\bmm + \bmm_K} \ba^* \cdot J\bX} 
\end{align*}
and
\begin{align*}
\lAngle f, (-i\nabla_\bx+\bK) g \rAngle^{+-}\pa{-\tfrac{3}{2}J\bX} =  \sum_{\bmm \in \Z^2} \pa{\bmm + \bmm_K}\ba^* \; C_\bmm e^{i 3 \pa{\bmm + \bmm_K} \ba^* \cdot \bX}
\end{align*}
so
\begin{align*}
\boxed{\lAngle f, (-i\nabla_\bx+\bK) g \rAngle^{+-}\pa{-\tfrac{3}{2}J\bX} = \f {1}3 \sum_{\bmm \in \Z^2} \bmm \ba^* \; C_{\f{\bmm - 3\bmm_K}{3}} e^{i \bmm \ba^* \cdot \bX} }
\end{align*}
so we can implement
\begin{align*}
\bm{\cA}_{j,j'}\pa{-\tfrac{3}{2}J\bX} = \lAngle u_j, (-i\nabla_\bx+\bK) u_{j'} \rAngle^{+-}\pa{-\tfrac{3}{2}J\bX}
\end{align*}

\subsection{$\bbW$'s $V\ind{int}$ term}%
\label{sub:_bbw_s_vint_term}
We write $V\ind{int}(z) = \sum_{m_z \in \Z} \widehat{V}\ind{int}^{m_z} e^{i \f{2\pi}L m_z z}$ hence
\begin{align*}
	\ps{u_j, V\ind{int} u_{j'}} &= \sum_{\substack{\bmm \in \Z^2 \\ m_z,m_z',M_z \in \Z}} \pa{\overline{\widehat{u}}_j}_{\bmm,m_z} \pa{\widehat{u}_{j'}}_{\bmm,m_z'}\pa{\widehat{V\ind{int}}}_{M_z} \int_z e^{iz \f{2\pi}{L} \pa{M_z + m_z'-m_z}} \\
& = L \sum_{\substack{\bmm \in \Z^2 \\ m_z,m_z' \in \Z}} \pa{\overline{\widehat{u}}_j}_{\bmm,m_z} \pa{\widehat{u}_{j'}}_{\bmm,m_z'}\pa{\widehat{V\ind{int}}}_{m_z-m_z'} 
\end{align*}
and the matrix $M_{j,j'} := \ps{u_j, V\ind{int} u_{j'}}$ is such that $M^* = M$ and $M_{11} = M_{22}$.





\section{BM configuration}%
\label{sec:bm_configuration}


From \cite{BecEmbWitZwo21}, the BM Hamiltonian is
\begin{align*}
	H = \mat{-i\sigma \na & T^c(x) \\ T^c(x)^* & -i \sigma \na},
\end{align*}
where
\begin{align*}
\boxed{T_1 = \mat{w_0 & w_1 \\ w_1 & w_0}, \quad  T_2 = \mat{w_0 &  w_1e^{-i\phi} \\  w_1e^{i\phi} & w_0}, \quad T_3 = \mat{w_0 &  w_1e^{i\phi} \\  w_1e^{-i\phi} & w_0}}
\end{align*}
and where, for $x \in \R^2$,
\begin{align*}
T^c(x) := \sum_{j=1}^3 T_j e^{-iq^c_j \cdot x} = \sum_{j=1}^3 T_j e^{iq_j a^*\cdot x}, \qquad \widehat{T}_p = \sum_{j=1}^{3} T_j \delta_{p,q_j^c}
\end{align*}
and
\begin{multline*}
q^c_1 = \f{4\pi}{a\sqrt{3}} \mat{1 \\ 0} = a_1^* + a_2^*, \\
q^c_2 = \f{4\pi}{a\sqrt{3}} \mat{-\f 12 \\ \f{\sqrt{3}}{2}} = -a_2^*, \qquad q^c_3 = \f{4\pi}{a\sqrt{3}} \mat{-\f 12 \\ -\f{\sqrt{3}}{2}} = -a_1^*,
\end{multline*}
where we took rotated $q_j^c$'s by $J$ with respect to \cite{BecEmbWitZwo21}, and with a rescaling of $\f{4\pi}{a\sqrt{3}}$.

We define the reduced dual vectors $q_j := -\cM^* q^c_j /2\pi$ so
\begin{align*}
T(x) = T^c(\cM x) = \sum_{j=1}^3 T_j e^{-ix\cdot \cM^* q_j^c}= \sum_{j=1}^3 T_j e^{i2\pi x\cdot q_j}
\end{align*}
and we compute
\begin{align*}
\boxed{q_1 = \mat{1 \\ 1}, \qquad q_{2} = \mat{0 \\ -1}, \qquad q_{3} = \mat{-1 \\ 0}}
\end{align*}
Or
\begin{align*}
\boxed{T(x) = \sum_{j=1}^3 T_j e^{i q_j a^* \cdot x}}
\end{align*}

Since $T_j^* = T_j$, then $T(-x)^* = T(x)$




\section{Operators in basis}%
\label{sec:operators_in_basis}


\subsection{Goal}%
\label{sub:goal}
Our goal is to study the eigenvalue equation
\begin{align*}
\cH \p = \cS \p
\end{align*}
where $\cS$ is Hermitian and positive and
\begin{align*}
\cH := \f{1}{\ep_\theta} \cV + c_\theta T + \ep_\theta T^{(1)}
\end{align*}
where
\begin{align*}
T &:=  v_{\rm F} \left( \begin{array}{cc} \bm\sigma \cdot (-i \nabla)  &  \bm{\mathcal A} \cdot (-i\nabla)   \\  \bm{\mathcal A}^* \cdot (-i\nabla) &  \bm\sigma \cdot (-i \nabla)  \end{array} \right),  \\
T^{(1)} &:= v_{\rm F} \left( \begin{array}{cc}  -  \bm\sigma \cdot J (-i \nabla) - \frac 12  \Delta &  \bm{\mathcal A} \cdot J (-i\nabla) - \frac 12  \Sigma \Delta  \\  \bm{\mathcal A}^*  \cdot J (-i\nabla) - \frac 12 \Sigma^* \Delta &  \bm\sigma \cdot J (-i \nabla) - \frac 12   \Delta \end{array} \right), \\
\cV &:=  \left( \begin{array}{cc}  \bbW &   \bbV \\   \bbV^* &   \bbW \end{array} \right),
\end{align*}
and their Bloch transform becomes
\begin{align*}
T_k &:=  v_{\rm F} \left( \begin{array}{cc} \bm\sigma \cdot (-i\na +k)  &  \bm{\mathcal A} \cdot (-i\na +k)   \\  \bm{\mathcal A}^* \cdot (-i\na +k) &  \bm\sigma \cdot (-i\na +k)  \end{array} \right),  \\
\hspace{-5cm} T_k^{(1)} &:= v_{\rm F} \left( \begin{array}{cc}  -  \bm\sigma \cdot J (-i\na +k) +\f 12  (-i\na +k)^2 &  \bm{\mathcal A} \cdot J (-i\na +k) +\f 12  \Sigma (-i\na +k)^2  \\  \bm{\mathcal A}^*  \cdot J (-i\na +k) +\f 12 \Sigma^* (-i\na +k)^2 &  \bm\sigma \cdot J (-i\na +k) +\f 12   (-i\na +k)^2 \end{array} \right)
\end{align*}
and we want the middle of the spectrum of
\begin{align*}
\cH_k := \cS^{-\f 12} \pa{\f{1}{\ep_\theta} \cV + c_\theta T_k + \ep_\theta T^{(1)}_k}\cS^{-\f 12}
\end{align*}


\subsection{Basis}%
\label{sub:basis}

We define $e_m := \f{1}{\sqrt{\ab{\Omega}}} e^{i m a^* \cdot x}$, and
\begin{align*}
e_{\alpha,m} := e_\alpha \otimes e_m = e_\alpha \f{e^{ima^*\cdot x}}{\sqrt{\ab{\Omega}}}, \qquad \text{where } e_1 := \mat{1 \\ 0 \\ 0 \\ 0},\dots
\end{align*}

\subsection{Multiplication-derivation operators}%
\label{sub:magnetic_term}

For $A = (A_1,A_2)$ and $A_j = \sum_\ell \pa{\widehat{A_j}}_\ell e^{i \ell a^*\cdot x}$, we have
\begin{multline*}
\ps{e_n, A \cdot (-i\na +k) e_m} = \sum_{\ell} \pa{\widehat{A_1}}_\ell \pa{ma^* + k}_1\ps{e_n, e^{i\ell a^*\cdot x} e_m} \\
+ \pa{\widehat{A_2}}_\ell \pa{ma^* + k}_2\ps{e_n, e^{i\ell a^*\cdot x} e_m} \\
= \pa{\widehat{A_1}}_{n-m} \pa{ma^* + k}_1 + \pa{\widehat{A_2}}_{n-m} \pa{ma^* + k}_2 = \widehat{A}_{n-m} \cdot \pa{ma^*+k}
\end{multline*}

For $V = \sum_\ell \widehat{V}_\ell e^{i\ell a^*x}$, we have
\begin{align*}
\ps{e_n,V (-i\na + k)^2 e_m} =  \pa{ma^*+k}^2\widehat{V}_{n-m}
\end{align*}
\subsection{On-diagonal potential}%
\label{sub:on_diagonal_potential}



For a general $W^\pm = \sum_m W^\pm_m e^{im a^* \cdot x}$, we have
\begin{align*}
	\ps{e_{\alpha,n},\mat{W^+ & 0 \\ 0 & W^-} e_{\beta,m}} = \delta_{\alpha \in \acs{1,2}}^{\beta \in \acs{1,2}}\pa{W^+_{n-m}}_{\alpha_1 \beta_1} + \delta_{\alpha \in \acs{3,4}}^{\beta \in \acs{3,4}}\pa{W^-_{n-m}}_{\alpha_2 \beta_2}
\end{align*}

\subsection{Off-diagonal potential}%
\label{sub:off_diagonal_potential}



For a general $V = \sum_m V_m e^{im a^* \cdot x}$, we have $V^* = \sum_m V_m^* e^{-im a^* \cdot x}$ and
\begin{align*}
	M_{IJ} & := \ps{e_{\alpha,n}, \mat{0 & V \\ V^* & 0} e_{\beta,m}} \\
	       &= \sum_{k} \pa{\delta_{\alpha \in \acs{1,2}}^{\beta \in \acs{3,4}} \delta_{m+k-n} \ps{e_{\alpha_1},V_k e_{\beta_2}} + \delta_{\alpha \in \acs{3,4}}^{\beta \in \acs{1,2}}\delta_{m-k-n} \ps{e_{\alpha_2},V_k^* e_{\beta_1}}} \\
	       &=   \delta_{\alpha \in \acs{1,2}}^{\beta \in \acs{3,4}}\ps{e_{\alpha_1},V_{n-m} e_{\beta_2}} + \delta_{\alpha \in \acs{3,4}}^{\beta \in \acs{1,2}}\ps{e_{\alpha_2},V_{m-n}^* e_{\beta_1}} \\
	       &=     \delta_{\alpha \in \acs{1,2}}^{\beta \in \acs{3,4}}\pa{V_{n-m}}_{\alpha_1 \beta_2} + \delta_{\alpha \in \acs{3,4}}^{\beta \in \acs{1,2}} \overline{\pa{V_{m-n}}_{\beta_1\alpha_2} }
\end{align*}
and $M$ is also Hermitian.

\subsection{Off-diagonal magnetic term}%
\label{sub:off_diagonal_magnetic_term}



For a general $A = \mat{A_1 \\ A_2}$, $A_j = \sum_\ell  \pa{A_j}_\ell  e^{i\ell a^*\cdot x}$, we have $A_j^* = \sum_\ell  \pa{A_j}^*_\ell  e^{-i\ell a^*\cdot x}$ and we compute
\begin{multline*}
\ps{e_{\alpha,n}, \mat{0 & A \cdot \pa{-i\na +k} \\A^* \cdot \pa{-i\na +k}  & 0} e_{\beta,m}} \\
= \delta_{\alpha \in \acs{1,2}}^{\beta \in \acs{3,4}} \pa{ \pa{ma^* +k}_1 \pa{\pa{A_1}_{n-m}}_{\alpha_1 \beta_2} + \pa{ma^* +k}_2 \pa{\pa{A_2}_{n-m}}_{\alpha_1 \beta_2}}\\
+ \delta_{\alpha \in \acs{3,4}}^{\beta \in \acs{1,2}} \pa{ \pa{ma^* +k}_1 \overline{\pa{\pa{A_1}_{m-n}}_{\beta_1 \alpha_2}} + \pa{ma^* +k}_2 \overline{\pa{\pa{A_2}_{m-n}}_{\beta_1 \alpha_2}}}
\end{multline*}


\subsection{Dirac operator}%
\label{sub:dirac_operator}

We have 
\begin{align*}
	\sigma \cdot \pa{-i\na + k} &= \sigma_1 \pa{-i\partial_1 + k_1} + \sigma_2 \pa{-i\partial_2 + k_2} \\
				    & = \mat{0 & -i\pa{\partial_1 -i\partial_2} + \overline{k_\C} \\ -i\pa{\partial_1 +i\partial_2} + k_\C & 0}
\end{align*}
 where
\begin{align*}
	\sigma_1 = \mat{0 & 1 \\ 1 & 0} \qquad \sigma_2 = \mat{0 & -i \\ i & 0}
\end{align*}
so, with $k_\C := k_1 +ik_2$,
\begin{align*}
\sigma \cdot \pa{-i\na + k} \mat{1 \\ 0} e_m = \pa{ma^* + k}_\C \mat{0 \\ 1} e_m \\
\sigma \cdot \pa{-i\na + k} \mat{0 \\ 1} e_m = \overline{\pa{ma^* + k}_\C} \mat{1 \\ 0} e_m
\end{align*}


Then
\begin{align*}
	\mat{\sigma \cdot \pa{-i\na + k} & 0 \\ 0 & \sigma \cdot \pa{-i\na + k}} e_{1,m} &= \pa{ma^* + k}_\C  e_{2,m} \\
	\mat{\sigma \cdot \pa{-i\na + k} & 0 \\ 0 & \sigma \cdot \pa{-i\na + k}} e_{2,m} &= \overline{\pa{ma^* + k}_\C}  e_{1,m} \\
	\mat{\sigma \cdot \pa{-i\na + k} & 0 \\ 0 & \sigma \cdot \pa{-i\na + k}} e_{3,m} &= \pa{ma^* + k}_\C  e_{4,m} \\
	\mat{\sigma \cdot \pa{-i\na + k} & 0 \\ 0 & \sigma \cdot \pa{-i\na + k}} e_{4,m} &= \overline{\pa{ma^* + k}_\C}  e_{3,m}
\end{align*}





We know that $e^{-ikx} \pa{-i\na} e^{ikx} \cdot = -i\na + k$ hence
\begin{align*}
e^{-ikx} \pa{-\f 12 \Delta} e^{ikx} \cdot = \f 12 \pa{-i\na + k}^2
\end{align*}
and with $f(x) = \sum_m \widehat{f}_m e^{ima^*x}$
\begin{align*}
\pa{-i\na +k} f = \sum_m  \pa{ma^* +k} \widehat{f}_m e^{ima^*x},
\end{align*}
so
\begin{align*}
\f 12 \pa{-i\na +k}^2 f = \sum_m  \f 12 \pa{ma^* +k}^2 \widehat{f}_m e^{ima^*x}
\end{align*}

We have
\begin{align*}
\ps{e_{\alpha,n}, \f 12 \pa{-i\na +k}^2 e_{\beta,m}} = \f 12 \pa{ma^* + k}^2 \delta_{\alpha,\beta} \delta_{m-n}
\end{align*}

We have
\begin{align*}
	\sigma \cdot k = \mat{0 & \overline{k_\C} \\ k_\C & 0}, \qquad \pa{J k}_\C = i k_\C, \qquad \sigma \cdot J k = \mat{0 & -i \overline{k_\C} \\ i k_\C & 0}
\end{align*}
so
\begin{align*}
	\mat{-\sigma \cdot J \pa{-i\na + k} & 0 \\ 0 & \sigma \cdot J \pa{-i\na + k}} e_{1,m} &= -i\pa{ma^* + k}_\C  e_{2,m} \\
	\mat{-\sigma \cdot J \pa{-i\na + k} & 0 \\ 0 & \sigma \cdot J \pa{-i\na + k}} e_{2,m} &= i \; \overline{\pa{ma^* + k}_\C} \; e_{1,m} \\
	\mat{-\sigma \cdot J \pa{-i\na + k} & 0 \\ 0 & \sigma \cdot J \pa{-i\na + k}} e_{3,m} &= i\pa{ma^* + k}_\C  e_{4,m} \\
	\mat{-\sigma \cdot J \pa{-i\na + k} & 0 \\ 0 & \sigma \cdot J \pa{-i\na + k}} e_{4,m} &= -i \; \overline{\pa{ma^* + k}_\C} \; e_{3,m}
\end{align*}

For a general $V = \sum_m \widehat{V}_m e^{ima^*\cdot x}$, we have $V^* = \sum_m \widehat{V}^*_m e^{-ima^*\cdot x}$ and we compute
\begin{multline*}
\ps{e_{\alpha,n}, \mat{0 & V  \pa{-i\na +k}^2 \\V^*  \pa{-i\na +k}^2  & 0} e_{\beta,m}} \\
=  \pa{ma^* +k}^2 \pa{\delta_{\alpha \in \acs{1,2}}^{\beta \in \acs{3,4}} \pa{\widehat{V}_{n-m}}_{\alpha_1 \beta_2} + \delta_{\alpha \in \acs{3,4}}^{\beta \in \acs{1,2}}   \overline{\pa{\widehat{V}_{m-n}}_{\beta_1 \alpha_2}}}
\end{multline*}


\section{Symmetries}%
\label{sec:symmetries}


\subsection{Particle-hole}%
\label{sub:particle_hole}

We define
\begin{align*}
\cS u(x) := i \mat{0 & -\1_{2\times 2} \\ \1_{2\times 2} & 0} u(-x)
\end{align*}
We have
\begin{align*}
\cS \mat{0 & B \\ B^* & 0} \cS = -\mat{0 & B^*(-x) \\ B(-x) & 0}
\end{align*}
We have $T(-x)^* = T(x)$ hence we should have that
\begin{align*}
\cS H \cS = - H
\end{align*}

We compute
\begin{align*}
	\cS_{IJ} &= \ps{e_{\alpha,n},\cS e_{\beta,m}} = i \ps{e_{\alpha,n}, \mat{-e_{\beta_2,-m} \\ e_{\beta_1,-m}}} \\
&= i \delta_{m+n} \pa{\delta_{\alpha \in \acs{3,4}}^{\beta \in \acs{1,2}} \delta_{\beta_1-\alpha_2} -\delta_{\alpha \in \acs{1,2}}^{\beta \in \acs{3,4}}\delta_{\beta_2-\alpha_1}}
\end{align*}

\subsection{Mirror}%
\label{sub:mirror}

First, for any function $B$, we have $\sigma_1 B^* \sigma_1 = \mat{\overline{B_{22}} & \overline{B_{12}} \\ \overline{B_{21}} & \overline{B_{11}}}$.


The mirror operator for the BM Hamiltonian is
\begin{align*}
	\cM u(\bX) := \mat{0 & \sigma_1 \\ \sigma_1 & 0} u(\overline{\bX})
\end{align*}
where $\overline{\bX} := (X_1,-X_2) =: M \bX$, it satisfies $\cM = \cM^{-1} = \cM^*$.


Next,
\begin{align*}
	\cM \mat{0 & B(\bX) \\ B(\bX)^* & 0} \cM = \mat{0 & \sigma_1 B^*(\overline{\bX}) \sigma_1 \\ \sigma_1 B(\overline{\bX}) \sigma_1 & 0}%= \mat{0 & \sigma_1 B(\bX)^* \sigma_1 \\ \sigma_1 B(\bX) \sigma_1 & 0}
\end{align*}

In cartesian coordinates, we have 
\begin{align*}
T(M \bX) = \sum_{j=1}^3 T_j e^{i  x \cdot M^*q^c_j} = \sum_{j=1}^3 T_j e^{i x \cdot Mq^c_j}
\end{align*}
because $M^* = M$. But
\begin{multline*}
	\sigma_1 T^*(M \bX) \sigma_1 = \mat{ w_0 \pa{\sum_{j=1}^3 e^{i x \cdot Mq_j}} & w_1\pa{e^{ix \cdot M q_1} + e^{i\phi} e^{ix Mq_2} + e^{i2\phi} e^{ix \cdot Mq_3}} \\ \cdot & \cdot} \\
	= \mat{ w_0 \pa{\sum_{j=1}^3 e^{i x \cdot q_j}} & w_1\pa{e^{ix \cdot q_1} + e^{-i\phi} e^{ix q_2} + e^{-i2\phi} e^{ix \cdot q_3}} \\ \cdot & \cdot} = T(\bX)
\end{multline*}
where we used that $M q_1^c = q_1^c$, $M q_2^c = q_3^c$ and $M q_3^c = q_2^c$.

We search the action on reduced Fourier coefficients. We have
\begin{align*}
f(M x) = \sum_m e^{i x \cdot M\pa{m a^*}} = \sum_m e^{i x \cdot \pa{M^r m}a^*}
\end{align*}
where $M = \mat{1 & 0 \\ 0 & -1}$,
\begin{align*}
M^r = S^{-1} M S = \cM^* M \pa{\cM^*}^{-1} = \sigma_1
\end{align*}


\section{Change of basis for getting $\Phi_j \in L^2_{\tau,\overline{\tau}}$}%
\label{sub:change_of_basis_for_getting_l_2__tau_tau}
Numerically, DFTK gives 
\begin{align*}
\phi, \p \in \Ker \pa{\cR_{\f{2\pi}{3}}-\tau} + \Ker\pa{\cR_{\f{2\pi}{3}}-\overline{\tau}}
\end{align*}
but we want to separate the spaces and obtain $\phi_1 \in \Ker \pa{\cR_{\f{2\pi}{3}}-\tau}$ so that $\phi_2(x,z) := \overline{\phi_1}(-x,z) \in \Ker\pa{\cR_{\f{2\pi}{3}}-\overline{\tau}}$, which existence is ensured by \cite{FefWei12}.

First we define
\begin{align*}
c := \nor{\pa{\cR_{\f{2\pi}{3}} -\tau}\phi_a}{L^2}^2, \qquad s := \ps{\pa{\cR_{\f{2\pi}{3}} -\tau}\phi_a,\pa{\cR_{\f{2\pi}{3}} -\tau}\phi_b}.
\end{align*}
Then we parametrize
\begin{align*}
\phi_1 = e^{i \alpha} \pa{\f s{\ab{s}} \cos \theta \phi_a + e^{i\beta} \sin \theta \phi_b}
\end{align*}
and we want $\pa{\cR_{\f{2\pi}{3}} -\tau}\phi_1 = 0$ which is equivalent to
\begin{align*}
\f s{\ab{s}}\cos \theta \pa{\cR_{\f{2\pi}{3}} -\tau}\phi_a + e^{i\beta} \sin \theta \pa{\cR_{\f{2\pi}{3}} -\tau}\phi_b =0
\end{align*}
and we take the scalar product with $\pa{\cR_{\f{2\pi}{3}} -\tau}\phi_a$ so that
\begin{align*}
\f c{\ab{s}}\cos \theta  + e^{i\beta} \sin \theta =0
\end{align*}
Now we necessarily have $e^{i\beta} = \pm$ so $\cos \theta = \mp \f {\ab{s}}c \sin \theta$ and finally using $\cos^2 + \sin^2 = 1$,
\begin{align*}
\ab{\cos \theta} = \f 1{\sqrt{1+ \pa{\f c{\ab{s}}}^2}}, \qquad \ab{\sin \theta} = \f 1{\sqrt{1+ \pa{\f {\ab{s}}c}^2}},
\end{align*}
and also choosing $\alpha = 0$ if $\cos \theta \ge 0$ and $\pi$ otherwise, which does not change anything, we have
\begin{align*}
\phi_1 = \f{s}{\ab{s}}\f 1{\sqrt{1+ \pa{\f c{\ab{s}}}^2}} \phi_a \pm \f 1{\sqrt{1+ \pa{\f {\ab{s}}c}^2}} \phi_b
\end{align*}
and $\phi_2(x) = \overline{\phi_1(-x)}$. By multiplying by $e^{-iKx}$, we also obtain
\begin{align*}
\boxed{u_1 = \f{s}{\ab{s}}\f 1{\sqrt{1+ \pa{\f c{\ab{s}}}^2}} u_a \pm \f 1{\sqrt{1+ \pa{\f {\ab{s}}c}^2}} u_b}
\end{align*}
and $u_2(x) = \overline{u_1(-x)}$.


\bibliographystyle{siam}
\bibliography{../../../biblio}
\end{document}
